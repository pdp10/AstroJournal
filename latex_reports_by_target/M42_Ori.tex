{\bf M42, Ori, CL+Neb}:
\begin{itemize}
\item xx/xx/1998 Mar 1998 to Jan 2015, Venice, Lorenzago (IT). Newcastle, Luton Devon, Cambridge, (UK). 1 to 3, 3 to 5. Celestron Newton 114mm F8; Binoculars 15x70; Tele Vue 60 F6, 36x; 15x, 18x, 30x, 36x, 51x, 72x. Orion nebula. C114F8, Venice (IT); TV60F6, Newcastle, Cambridge (UK).
\item 23/02/2015 19:00-21:00, Cambridge, UK. 2 - Slight undulations, 3 - Somewhat clear. Tele Vue 60 F6, 15x, 18x. Great Orion Nebula. M42 benefits from both UHC and OIII filters, but in different way. The OIII shows a sublime image where the border between the nebula and sky background really emerges. The same can be said about the North part (that one linked to M43). In the centre of the nebula, some 'waves' were also visible. It is a super target to my eye. The UHC shows a much larger extension for this nebula and this is amazing with a wide field telescope. Small fine details visible within the nebula with the OIII are less obvious with the UHC, but the nebula just appears as massive globally and faint details on the outside borders are accessible as pure diffuse bright areas. 
\item 22/03/2015 19:00-22:00, Cambridge, UK. 2 - Slight undulations, 3 - Somewhat clear. Tele Vue 60 F6, 15x + OIII, 51x. Well balanced contrast at 15x with OIII. 51x shows trapezium
\end{itemize}
