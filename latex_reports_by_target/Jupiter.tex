{\bf Jupiter, , Planet}:
\begin{itemize}
\item xx/xx/1998 Mar 1998 to Jan 2015, Venice, Lorenzago (IT). Newcastle, Luton Devon, Cambridge, (UK). 1 to 3, 3 to 5. Celestron Newton 114mm F8; Binoculars 15x70; Tele Vue 60 F6, 36x, 90x, 180x; 72x, 103x, 144x. C114F8, Italy (IT); B15x70, TV60F6, Newcastle, Cambridge (UK).
\item 23/02/2015 19:00-21:00, Cambridge, UK. 2 - Slight undulations, 3 - Somewhat clear. Tele Vue 60 F6, 103x, 144x. A bit of wind, but the image stays crisp at high magnifications. No aberration.
\item 22/03/2015 19:00-22:00, Cambridge, UK. 2 - Slight undulations, 3 - Somewhat clear. Tele Vue 60 F6, 144x. Order: Europa, Callisto, Jupiter, Io, Ganymede. Two belts very visible. The lower one was visible on the left (refractor). On the right the great red spot was detectable. Very minor belts north and south. 
\item 25/03/2015 21:00-22:45, Cambridge, UK. 2 - Slight undulations, 3 - Somewhat clear. Tele Vue 60 F6, 72x. Quick observation. Two belts and four satellite were visible. 
\item 06/04/2015 21:00-22:45, Cambridge, UK. 2 - Slight undulations, 3 - Somewhat clear. Tele Vue 60 F6, 103x. Two belts clearly visible and a faint one in the South hemisphere was detectable. All four satellite were visible. Io and Europa were very tight at East of the planet.
\item 09/04/2015 21:20-22:45, Cambridge, UK. 2 - Slight undulations, 2 - Poor. Tele Vue 60 F6, 15x; 103x +/- VPF. At the eyepiece from right to left: Callisto, Io, Jupiter, Europa and Ganymede. This evening I decided to test my new eyepiece (Nagler 3.5mm). Due to the lack of transparency, I only tested this on Jupiter. This was the first time I observed at 103x without using a Bresser 2x SA. The difference was quite substantial. I had the impression that the Nagler 7mm with Bresser 2x SA was more colour corrected than the Nagler 3.5mm only at the edge (last 10\% before the field stop). This might have been caused by the presence of light fog though, instead of the eyepiece. I will test this again. On the other hand, the lack of the Bresser 2x SA (4 lens less) improved transparency, and this was detectable. With a Nagler 7mm and Bresser 2x SA, I am able to see a bit more than the two main belts only when the seeing is quite good. Tonight, although the seeing was acceptable, but the sky was quite foggy. The main two belts (North and South Equatorial Belts) were visible and other two belts at the poles were easily detectable (North Polar Region, S.S. Temperate Belt). In the North and South Equatorial Belts, some shades were also detectable. No direction was visible but it was possible to see that the borders and belt colours were rough and not homogeneous. This was particularly true for the North Equatorial Belt. No GRS was detectable. The use of a single or double polarizing filter did not improve image quality. The whole image only appeared too dark and the minute details previously described were lost. Possibly, the VPF is more appropriate for brighter objects (e.g. the Sun and the Moon) or Jupiter during sunset or dawn. 
\item 14/04/2015b 21:30-23:20, Cambridge, UK. 2 - Slight undulations, 5 - Clear. Tele Vue 60 F6, 103x, 72x. Transit of Ganymede on Jupiter. Little black dot on the Equatorial zone. All the other three main satellites were well distict on right.
\item 30/04/2015 22:00-23:00, Cambridge, UK. 2 - Slight undulations, 5 - Clear. Tele Vue 60 F6, 103x. At 103x Jupiter showed 4 moons and 4 belts. No specific events were visible this evening.
\item 12/05/2015 21:00-23:45, Cambridge, UK. 2 - Slight undulations, 5 - Clear. Tele Vue 60 F6, 103x, 206x. Observed in the twilight. Still visible at 206x with some detail but the new tripod is not up to this sort of magnifications. To be fair, the new tripod was fine at 103x but only when there was no wind. 3 belts and 4 moons visible. It would be interesting to try 206x with my solid tripod.
\item 13/05/2015 21:00-0:00, Cambridge, UK. 1 - Perfect seeing, 5 - Clear. Tele Vue 60 F6, 103x, 206x. Observed in the twilight. The idea started as a joke because I thought the image would have been too dark for discerning any detail. Instead, it was possible to perceive a little amount of shades on the two major belts of the planet. The boundaries of the other two less visible belts (North and South hemisphere, respectively) were also there. At 103x I was able to see the boundaries of these two belts on the 'equator side', but not on the 'pole side'. At 206x these were noticeable. 4 moons were detectable and one was just about to get closer to Jupiter. I agree with Gerry (sgl: jetstream) that watching Jupiter in twilight shows more contrast. I was also able to see some red-ish colour on the major two belts, which instead is less noticeable when watching Jupiter in the dark. Looking at a bright source before watching the planet did not help me instead. I found I had more difficulty to notice details. Although the exit pupil was only 0.3mm, floaters did not cause me serious problems. Interestingly, I found floaters to be a problem when watching the Sun at 103x. Could these be related to overall image brightness? 
\item 20/05/2015 21:30-0:00, Cambridge, UK. 3 - Moderate seeing, 5 - Clear. Tele Vue 60 F6, 103x +/- SPF, 206x + SPF. Observed in the twilight. The SPF noticeably improved the view. Four belts and the transit of Callisto were easily visible at both 103x and 206x. The use of a SPF seemed to stabilise the image and improved contrast. A fair amount of shades were also perceptible on the main two belts. The transit appeared as a crisp black dot on the planet atmosphere. Without the SPF it was only possible to see the two main belts and no shade on them. They simply appeared as two thick lines across the planet. Interestingly the transit shadow appeared a tiny bit better without the filter. To me, using the SPF requires a bit of experience in order to rotate the eyepiece to gain the best contrast. However this is feasible.
\item 26/05/2015 21:15-0:00, Cambridge, UK. 2 - Slight undulations, 5 - Clear. Tele Vue 60 F6, 103x +/- SPF. Observed in civil twilight. The seeing was not enough good for pushing magnification beyond 103x. At 103x, two major belts and two moons were visible. I did not spend much on this target tonight because it was too windy when I observed it.
\item 03/06/2015 21:40-23:30, Cambridge, UK. 1 - Perfect seeing, 5 - Clear. Tele Vue 60 F6, 103x +/- SPF, 206x + SPF. Just a quick look until the sky became darker. No particular event tonight. It was very nice to see it. The two main belts revealed some subtle detail appearing like tiny shades. In particular these were more detectable in the North Equatorial Belt.  
\item 11/06/2015 22:00-0:00, Cambridge, UK. 3 - Moderate seeing, 3 - Somewhat clear. Tele Vue 60 F6, 103x, 206x. 206x was too much for Jupiter tonight. Mostly seen it at 103x. Three moons visible, whereas the fourth seemed behind the planet. North and South Equatorial Belts were visible. On the North Hemisphere another belt was also detectable. No GRS visible.
\item 15/06/2015 21:45-0:30, Cambridge, UK. 2 - Slight undulations, 5 - Clear. Tele Vue 60 F6, 72x, 144x +/- SPF. Jupiter was visible with North and South Equatorial Belts and four moons. No other detail was detectable. SPF did not help with the Vixen. The planet did not appear much crisp in contrast to Venus. Also here, I preferred the view of Jupiter at 72x (without Barlow). It is as if the barlow lens introduces some imperfections which remove the additional benefit of using a Vixen vs a Nagler. The same can be said for the SPF with the Vixen. Vixen alone gave the best views (without Barlow or SPF).
\item 23/06/2015 21:40-23:15, Cambridge, UK. 2 - Slight undulations, 5 - Clear. Tele Vue 60 F6, 72x, 103x. Observation at twilight. All four moons were visible tonight. At 72x 4 belts were detectable and colours were also visible. These appeared as brown-red for the belts and slightly darker white for the zones. At 103x the image was a bit degraded compared to 72x. I believe the Vixen SLV is just a tiny but noticeable bit better than the nagler 3.5mm
\end{itemize}
