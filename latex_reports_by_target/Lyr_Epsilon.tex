{\bf Epsilon, Lyr, Dbl star}:
\begin{itemize}
\item 03/06/2015 21:40-23:30, Cambridge, UK. 1 - Perfect seeing, 5 - Clear. Tele Vue 60 F6, 15x, 103x, 206x. The Double Double. Epsilon 1 and 2 were easily split at 15x. At 103x it was possible to detect that both Epsilon 1 and 2 are double stars themselves. At 206x this pair of tight double stars was visible although these double stars remained very close. Same colour.
\item 10/06/2015 22:00-0:00, Cambridge, UK. 3 - Moderate seeing, 5 - Clear. Tele Vue 60 F6, 51x. The Double Double. I could not split the two. The image suggested a possible elongation of the two stars, but this was not obvious. I would not have detected it if I had not known that they are doubles. I carefully focused inward and outward to reach the optimum, but this was not sufficient. 
\item 11/06/2015 22:00-0:00, Cambridge, UK. 3 - Moderate seeing, 3 - Somewhat clear. Tele Vue 60 F6, 103x. The Double Double. Just managed to see the two pairs, although the separation was not clear. They appeared just a tiny more than elongated stars. I am not sure, but I suspect this was more due to the Nagler 3.5mm. I will try with the Vixen 5mm next time, as generally this eyepiece delivers better views than the Naglers, on planets at least.
\item 15/06/2015 21:45-0:30, Cambridge, UK. 2 - Slight undulations, 5 - Clear. Tele Vue 60 F6, 72x. The Double Double. It was possible to see the two pairs at 72x, although to me this was not appreciable. The two pairs appeared a little bit more than elongated or just separated, but I much prefer when a double star is clearly and nicely separated. The two pairs were similarly separated. Possibly Epsilon1 (the North pair) was slightly more, but, if so, a tiny bit.
\end{itemize}
