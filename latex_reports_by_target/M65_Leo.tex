{\bf M65, Leo, Galaxy}:
\begin{itemize}
\item 22/03/2015 19:00-22:00, Cambridge, UK. 2 - Slight undulations, 3 - Somewhat clear. Tele Vue 60 F6, 15x. Leo triplet. From Chertan (theta Leo), use the star pointers HIP54688 and HIP54711 to reach eta Leo. Eta Leo forms a 90Deg triangle with HIP55033 and HIP55262. From the latter look at south slightly. Galaxy detectable as patches. M56 is elongated. Averted vision for 10min is required. Cover the other eye to relax the observing eye nerve.
\item 25/03/2015 21:00-22:45, Cambridge, UK. 2 - Slight undulations, 3 - Somewhat clear. Tele Vue 60 F6, 15x, 18x, 30x. Invisible. Sky not transparent enough. I think an exit pupil of 3.3mm is a good compromise between 4mm and 2mm. 2mm is too much for the TV60 on this targets.
\item 06/04/2015 21:00-22:45, Cambridge, UK. 2 - Slight undulations, 3 - Somewhat clear. Tele Vue 60 F6, 30x. Leo triplet. Elongated grey patch visible with averted vision. Shape of a cigar. At 30x, the patch is visible more easily than at 15x. The leo triplet is more easily detectable when the telescope is slightly moved. The patches will move accordingly.
\item 14/04/2015b 21:30-23:20, Cambridge, UK. 2 - Slight undulations, 5 - Clear. Tele Vue 60 F6, 15x, 18x. This object requires aperture and dark sky to be detected and viewed properly. Just very faint object visible through averted vision. An exit pupil of 3.3mm is better than 4.0mm. I wonder whether something between 2.5 and 2.0mm can improve this view even more.
\end{itemize}
