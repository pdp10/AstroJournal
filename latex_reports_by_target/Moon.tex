{\bf Moon, Satellite}:
\begin{itemize}
\item xx/xx/1998 Mar 1998 to Jan 2015, See location(s) later. Moderate to perfect seeing, Moderate to clear transparency. See telescope(s) later, 36x, 90x, 180x; 15x, 51x, 72x, 103x, 144x. C114F8, Italy (IT); B15x70, TV60F6, Newcastle, Cambridge (UK).
\item 24/03/2015 19:00-21:30, Cambridge, UK. 3 - Moderate seeing, 3 - Somewhat clear. Tele Vue 60 F6, 72x. Waxing crescent 25\%. Very crisp details. 
\item 30/04/2015 22:00-23:00, Cambridge, UK. 2 - Slight undulations, 5 - Clear. Tele Vue 60 F6, 51x, 103x. Moon phase 91\%. The moon at 103x did not need a filter. It was very crisp and showed details in the South hemisphere despite it was almost full. At 51x, the moon is simply scaled of a factor of 1/2, indicating that the Nagler 3.5mm behaves as a perfect 2x Nagler 7mm. It would be useful to have a Moon map to check the crater's names. 
\item 26/05/2015 21:15-0:00, Cambridge, UK. 2 - Slight undulations, 5 - Clear. Tele Vue 60 F6, 103x +/- SPF, 206x + SPF. Observed in the twilight. Visible almost 60\% of its phase. The SPF seems to stabilise the image if the seeing is not good. This is a lovely target with the TV60, and keeps magnification pretty well. At 206x the moon surface appeared like a bubble at the poles due to the seeing, but there were moments in which it was possible to see a quasi stable image. Subtle details on the surface were observable as well as minute craters and shades on the ground. Interestingly, on the terminator mounts tips were illuminated whereas their bases were obscured. There is so much to see at 206x that one could spend the entire night observing our satellite! Montes Apenninus, Caucasus, and Alpes were incredible targets and appeared just beautiful. The crater Cassini and all the small nearby craters were spectacular. While I am not sure the SPF increased image contrast, I prefer the view with SPF as it seems that the image is just stabler at both 103x and 106x.
\item 03/06/2015 21:40-23:30, Cambridge, UK. 1 - Perfect seeing, 5 - Clear. Tele Vue 60 F6, 103x +/- SPF. Phase 96\%. No many detail were revealed. The moon is not really interesting when full. Craters and seas were detectable but not immersive. 
\item 23/06/2015 21:40-23:15, Cambridge, UK. 2 - Slight undulations, 5 - Clear. Tele Vue 60 F6, 28x, 72x, 103x, 206x. Observation at twilight. Waxing crescent at about 40\%. At 72x, the Moon was really beautiful and crisp. Very soft little clouds passed over the South hemisphere and the view was really suggestive. At 103x, some more detail were visible, although these were somehow lost at 206x due to the non perfect seeing which did not allow to get a perfect focus. At 28x, the Moon appeared as a lovely target floating on the sky. The ultra wide field of the Nagler 13mm really shows the Moon and the surrounding context.
\item 29/06/2015 21:30-22:00, Cambridge, UK. 2 - Slight undulations, 4 - Partly clear. Tele Vue 60 F6, 72x, 103x, 206x. Observation at twilight. Waxing Gibbous, 94\%. The moon was not very crisp tonight due to a little layer of high clouds caused by the high temperature during the day. I moved from Montes Apenninus to Copernicus. A small crater was visible inside, but many details on the circular border were not clear. Therefore I moved to Kepler as this was farther East hoping to improve the visible contrast. This showed a little shadow on one border. At 206x, from Kepler I moved North-East, following the crater chain formed by Kepler C, Marius D, F, A, C, and B. All these craters are relatively small and close to each other. Finally I moved North reaching Aristarchus which appeared beautiful. This white crater shows an impressive contrast and is close to a little half circle of hills at North-East and a dark crater (Herodotus) at East. Really nice view.
\item 30/06/2015 21:30-23:20, Cambridge, UK. 3 - Moderate seeing, 4 - Partly clear. Tele Vue 60 F6, 72x, 144x, 206x. Observation at twilight. Waxing Gibbous and phase 97\%. I observed Tycho, Copernicus and Kepler. 
\end{itemize}
