{\bf Moon, Leo, Satellite}:
\begin{itemize}
\item 26/05/2015 21:15-0:00, Cambridge, UK. 2 - Slight undulations, 5 - Clear. Tele Vue 60 F6, 103x +/- SPF, 206x + SPF. Observed in the twilight. Visible almost 60\% of its phase. The SPF seems to stabilise the image if the seeing is not good. This is a lovely target with the TV60, and keeps magnification pretty well. At 206x the moon surface appeared like a bubble at the poles due to the seeing, but there were moments in which it was possible to see a quasi stable image. Subtle details on the surface were observable as well as minute craters and shades on the ground. Interestingly, on the terminator mounts tips were illuminated whereas their bases were obscured. There is so much to see at 206x that one could spend the entire night observing our satellite! Montes Apenninus, Caucasus, and Alpes were incredible targets and appeared just beautiful. The crater Cassini and all the small nearby craters were spectacular. While I am not sure the SPF increased image contrast, I prefer the view with SPF as it seems that the image is just stabler at both 103x and 106x.
\end{itemize}
