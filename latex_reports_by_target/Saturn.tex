{\bf Saturn, Planet}:
\begin{itemize}
\item xx/xx/1998 Mar 1998 to Jan 2015, Venice, Lorenzago (IT). Newcastle, Luton Devon, Cambridge, (UK). 1 to 3, 3 to 5. Celestron Newton 114mm F8; Binoculars 15x70; Tele Vue 60 F6, 36x, 90x, 180x. C114F8, Venice (IT).
\item 12/05/2015 21:00-23:45, Cambridge, UK. 2 - Slight undulations, 5 - Clear. Tele Vue 60 F6, 103x. Very low on the horizon and therefore not the best moment for viewing this target. Despite this, rings and titan were visible. Neither the Cassini division nor belts were detectable.
\item 13/05/2015 21:00-0:00, Cambridge, UK. 1 - Perfect seeing, 5 - Clear. Tele Vue 60 F6, 103x. It was a bit higher than yesterday, but unfortunately, my telescope and eyepieces were soaked with humidity and could not really see this target after the first 5 min. Rings were clearly defined, and I believe the Cassini division could have been detectable.
\item 20/05/2015 21:30-0:00, Cambridge, UK. 3 - Moderate seeing, 5 - Clear. Tele Vue 60 F6, 103x, 206x. It is still fairly low on the horizon. Titan was visible. The Cassini division was not detectable, but it was possible to see a shade in the middle of the ring. At 206x the image was just degraded and difficult to focus. I have to wait for a higher position of the planet.
\item 10/06/2015 22:00-0:00, Cambridge, UK. 3 - Moderate seeing, 5 - Clear. Tele Vue 60 F6, 103x. Seen during civil twilight and later in the nautical twilight. Although the seeing was not great, Saturn appeared very crisp. The rings had a very nice inclination. The Cassini division was generally not detectable. For few seconds when the seeing stabilised, a hint of dimmer colour was visible on the external part of the rings. A nice belt was visible all the time in the North hemisphere (North Equatorial Belt) of the planet. Titan was visible too. The view was really nice generally. Possibly due to the seeing, but I preferred the view when the sky was darker. 
\item 11/06/2015 22:00-0:00, Cambridge, UK. 3 - Moderate seeing, 3 - Somewhat clear. Tele Vue 60 F6, 103x, 206x. Very nice view of Saturn tonight. At both 103x and 206x, the Cassini division was detectable when the sky appeared steady for few seconds. It appeared as a soft grey shade on the lateral parts of the rings. Possibly what I was seeing was the shade between the A and B rings. This was not always visible, but just for few seconds when the seeing was steady and no wind blew, the difference in colour intensity was noticeable. Titan was also visible on the South of the planet. It seemed a grey dot. The North Equatorial Belt on the planet appeared as a soft darker gradient compared to the planet equatorial zone. The North Polar Region was not clearly detectable.
\item 15/06/2015 21:45-0:30, Cambridge, UK. 2 - Slight undulations, 5 - Clear. Tele Vue 60 F6, 28x, 72x +/- SPF, 144x. Saturn was wonderful with the Vixen at 72x. The North Equatorial Belt was detectable particularly when in contrast with the Equatorial zone. The Cassini division was visible on the left and right parts of the rings when the planet was at the centre of the eyepiece. It appeared as a soft grey line which separated more dense rings (B rings) from lighter rings (A rings). The shadow of the planet on the ring or details on the polar region were not visible. Titan was also visible. A SPF did not help and actually degraded the image for Saturn with the Vixen. At 144x, the image degraded and was not as nice as at 72x. At 28x, the planet was very small, but the rings and the empty part between the planets and the rings were visible. Titan at South-West of the planet in the eyepiece was much brighter at this magnification (due to the larger exit pupil) and I felt a small faint dot was detectable at South-East of the planet in the eyepiece. This was closer to the planet than Titan. After checking Saturn's moons positions with Sky and Telescope software application, the only moon at that distance and position was Rhea. I am not sure I saw this moon of magnitude 10. It would be at the limit of my TV60. This dot was more visible with averted vision although it was also detectable via direct vision.
\item 23/06/2015 21:40-23:15, Cambridge, UK. 2 - Slight undulations, 5 - Clear. Tele Vue 60 F6, 18x, 28x, 72x, 103x. Observation at twilight. At 72x, Titan was clearly visible and appeared yellow-orange. The planet appeared yellow globally, whereas one the North Equatorial Belt was more orange. The Cassini division was detectable on the lateral parts of the rings and the Ring A was distinguishable from the Ring B, due to the difference colour intensity. At 103x, no additional detail was visible, but the Cassini division was still there on the lateral parts. At 28x, the rings were visible and well separated from the planet. Titan's colour was a bit more orange. It is really interesting that these colours appear much more evident when the sky is clear rather than dark. I believe this is due to the eye cones which are more active than the eye rods. At 18x, I could not really distinguish the rings from the planet, although the non spherical shape was observable. No additional moon was detectable at these low powers. 
\item 30/06/2015 21:30-23:20, Cambridge, UK. 3 - Moderate seeing, 4 - Partly clear. Tele Vue 60 F6, 72x. Observation at twilight. Due to the poor seeing, I did not push magnifications higher than 72x. Even at this zoom, the planet was not very crisp. 
\item 02/07/2015 21:50-0:00, Cambridge, UK. 2 - Slight undulations, 5 - Clear. Tele Vue 60 F6, 28x, 51x, 72x, 103x . Observation at twilight. At 28x I could spot the rings and Titan. At 51x Saturn was very crisp but not additional detail was detectable. At 72x, the North Equatorial Belt and the Cassini division on the lateral rings were visible. At 103x those gained details were somehow lost unfortunately. Saturn was lovely at 72x.
\item 05/07/2015 21:50-0:20, Cambridge, UK. 3 - Moderate seeing, 5 - Clear. Tele Vue 60 F6, 28x, 51x, 72x, 103x . Observation at twilight. At 72x, the North Equatorial Belt was easily visible. The Cassini division was detectable at the lateral sides of the rings, and the rings A and B were clearly distinct. Titan was visible and appeared like a small star. At 28x or 51x the planet looked crisper, but the NEB was not easily detectable. At 103x the Cassini division was not visible. The seeing was not good enough for higher power, unfortunately.
\end{itemize}
