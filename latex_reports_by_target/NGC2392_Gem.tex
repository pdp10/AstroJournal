{\bf NGC2392, Gem, Pln Neb}:
\begin{itemize}
\item xx/xx/1998 Mar 1998 to Jan 2015, Venice, Lorenzago (IT). Newcastle, Luton Devon, Cambridge, (UK). 1 to 3, 3 to 5. Celestron Newton 114mm F8; Binoculars 15x70; Tele Vue 60 F6, 15x, 51x, 72x. Eskimo Nebula. TV60F6, Cambridge (UK).
\item 25/03/2015 21:00-22:45, Cambridge, UK. 2 - Slight undulations, 3 - Somewhat clear. Tele Vue 60 F6, 15x, 51x +/- OIII, UHC, 72x. Eskimo nebula. From Wasat (Delta Gem) move east to 63 Gem. 63 Gem is the brightest star of a 'half moon' of 7 stars. The Eskimo nebula is next to the star HIP36370 (mag8.2), which is a bit isolated but very close to 63 on the opposite direction of Wasat. You can spot it at 15x without filters, but you see it only with averted vision. It appears as a very small patch next to the star. At 51x the nebula is visible as a grey little ball. The boundaries are obfuscated. An UHC filter helps increasing the contrast between the sky and the nebula. An OIII filter shows even more contrast, although I think an UHC filter is better at this exit pupil (1.2mm). Using these filters, the boundaries of the nebula appear much clearer although no structure is visible at this magnification. At 72x (and no filter) is still visible as a grey little ball. Boundaries are obfuscated.  
\item 30/04/2015 22:00-23:00, Cambridge, UK. 2 - Slight undulations, 5 - Clear. Tele Vue 60 F6, 15x, 51x. Eskimo nebula. At 15x it was detectable with averted vision. It was easily visible at 51x and appeared like a fuzzy blue/grey small patch next to the star. 
\end{itemize}
