{\bf M97, UMa, Pln Neb}:
\begin{itemize}
\item 20/05/2015 21:30-0:00, Cambridge, UK. 3 - Moderate seeing, 5 - Clear. Tele Vue 60 F6, 28x +/- OIII. Owl Nebula. Interesting target. Completely invisible without OIII filter. With the OIII, it emerges from the sky and the nearby stars. It is a quite large planetary nebula. No colour or shape was detectable, but it simply appeared as a grey bubble. At 15x + OIII was detectable, but was too small to see any major detail.
\item 26/05/2015 21:15-0:00, Cambridge, UK. 2 - Slight undulations, 5 - Clear. Tele Vue 60 F6, 28x +/- UHC. Owl Nebula. Invisible at 28x with or without UHC filter. This target requires an OIII filter for being detectable with small aperture telescopes. Consistently with what said for M57, the OIII filter is a better choice for planetary nebulae (and for extended nebulae where we want to maximise nebulae contrast).
\end{itemize}
