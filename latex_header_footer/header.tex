\documentclass[10pt,twoside,a4paper,english]{article}
\usepackage[a4paper,margin=1in,landscape]{geometry}
\usepackage[colorlinks=true,linkcolor={black},urlcolor={black}]{hyperref}
\usepackage{longtable}
\usepackage{pdflscape}
\title{Astronomy Observation Journal}
\author{Piero Dalle Pezze}
\date{\today}


\begin{document}

\maketitle
\thispagestyle{empty}

\begin{abstract}
This document contains my observation reports and the objects listed by catalogue that I have so far observed. At the end the complete Messier and Caldwell catalogues are also presented. This file was generated using Java software tool {\it AstroJournal} (\href{https://pdp10@bitbucket.org/pdp10/astrojournal.git}{https://pdp10@bitbucket.org/pdp10/astrojournal.git}) and {\it pdflatex} (\href{http://www.tug.org/texlive/}{http://www.tug.org/texlive/}). {\it AstroJournal} imports files containing astronomy observation reports and observed objects by catalogues. Once imported, it generates an integrated journal document in \LaTeX\ which is then exported in PDF using the utility {\it pdflatex}. {\it AstroJournal} is released under GPL v3 license.
\end{abstract}

\tableofcontents


\clearpage


%\footnotesize
\small
%\normalsize


\section{Legends}

\noindent 

\bigskip 
{\bf Seeing Scale (Antoniadi):}
\begin{enumerate}
\item Perfect seeing, without a quiver
\item Slight undulations, with moments of calm lasting several seconds
\item Moderate seeing, with larger air tremors
\item Poor seeing, with constant troublesome undulations
\item Very bad seeing, scarcely allowing the makings of a rough sketch
\end{enumerate}

\bigskip     
{\bf Transparency Scale (American Association of Amateur Astronomers):}
\begin{enumerate}
\item Do Not Observe: Completely cloudy or precipitating (Why are you out?)
\item Very Poor: Mostly Cloudy
\item Poor: Partly cloudy or heavy haze. 1 or 2 Little Dipper stars visible
\item Somewhat Clear: Cirrus or moderate haze. 3 or 4 Little Dipper stars visible
\item Partly Clear: Slight haze. 4 or 5 Little Dipper stars visible
\item Clear: No clouds. Milky Way visible with averted vision. 6 Little Dipper stars visible
\item Very Clear: Milky Way and M31 visible. Stars fainter than mag 6.0 are just seen and fainter parts of the Milky Way are more obvious 
\item Extremely Clear: overwhelming profusion of stars, Zodiacal light and the gegenschein form continuous band across the sky, the Milky Way is very wide and bright throughout
\end{enumerate}

\bigskip
{\bf Target Types:}
\begin{itemize}
\renewcommand\labelitemi{--}
\item SN Rem: Supernova Remnant
\item Neb: Nebula
\item Galaxy: Galaxy
\item CL+Neb: Cluster with Nebula
\item Opn CL: Open Cluster
\item Glob CL: Globular Cluster
\item Pln Neb: Planetary Nebula
\item Satellite: Our Satellite (Moon)
\item Planet: Solar System Planet
\item Star: Star
\item Dbl Star: Double Star
\item Mlt Star: Multiple Star
\item Asterism: Asterism
\end{itemize}

\clearpage


%\footnotesize
%\small
\normalsize
