\documentclass[10pt,twoside,a4paper]{article}
\usepackage[a4paper,margin=1in,landscape]{geometry}
\usepackage{color}
\usepackage{graphicx}
\usepackage{longtable}
\usepackage{pdflscape}
% DO NOT EDIT THIS FILE AS IT IS AUTO-GENERATED by the program AstroJournal. 
\title{Astronomy Observation Journal}
\author{Piero Dalle Pezze}
\date{\today}
\begin{document}
\maketitle
\begin{abstract}
This document contains a collection of observation reports. It is auto-generated by running Java program AstroJournal (https://pdp10@bitbucket.org/pdp10/astrojournal.git) and the utility pdflatex (or texi2pdf). AstroJournal generates Latex code from tab-separated value (tsv) files exported using a SpreadSheet (e.g. Google Drive SpreadSheet). This allows users to: a) edit their observation reports using a spreadsheet easily, b) obtain a complete and formatted report in Latex, and c) benefit from versioning (if using a versioning system such as Git). 
\end{abstract}


% Preamble. 
\newpage
\footnotesize

\noindent 
{\bf Atlases:}
\begin{enumerate}  
\item Deep Sky Hunter Star Atlas v2 (by Michael Vlasov)
\item Sky \& Telescope's Pocket Sky Atlas (by Roger W. Sinnott)
\item Carte du Ciel (software)
\item Stellarium on tablet (software)
\end{enumerate}
\bigskip 
{\bf Observing techniques for DSO:}
\begin{enumerate}
\item Eye adaptation at the eyepiece for 10min at least 
\item Averted vision 
\item Cover the other eye to relax the observing eye nerve 
\item Know exact target position (precise star hopping) 
\end{enumerate} 
\bigskip 
{\bf Antoniadi Scale:}
\begin{enumerate}
\item Perfect seeing, without a quiver.
\item Slight undulations, with moments of calm lasting several seconds.
\item Moderate seeing, with larger air tremors.
\item Poor seeing, with constant troublesome undulations.
\item Very bad seeing, scarcely allowing the makings of a rough sketch.
\end{enumerate}
\bigskip 
{\bf Transparency Scale (American Association of Amateur Astronomers):}
\begin{enumerate}
\item Do Not Observe - Completely cloudy or precipitating. (Why are you out?)
\item Very Poor - Mostly Cloudy. 
\item Poor - Partly cloudy or heavy haze. 1 or 2 Little Dipper stars visible. 
\item Somewhat Clear - Cirrus or moderate haze. 3 or 4 Little Dipper stars visible. 
\item Partly Clear - Slight haze. 4 or 5 Little Dipper stars visible. 
\item Clear - No clouds. Milky Way visible with averted vision. 6 Little Dipper stars visible. 
\item Very Clear - Milky Way and M31 visible. Stars fainter than mag 6.0 are just seen and fainter parts of the Milky Way are more obvious 
\item Extremely Clear - overwhelming profusion of stars, Zodiacal light and the gegenschein form continuous band across the sky, the Milky Way is very wide and bright throughout
\end{enumerate}
\newpage

% General observation data
\begin{tabular}{ p{0.9in} p{1.3in} p{1.2in} p{5.2in}}
{\bf Date:} & 06/06/2015 & {\bf Telescopes:} & Tele Vue 60 F6 \\ 
{\bf Time:} & 15:00-18:00 & {\bf Eyepieces:} & Nagler 7mm, Vixen 5mm, Nagler 3.5mm \\ 
{\bf Location:} & Cambridge, UK & {\bf Power, EP, FOV:} & 51x, 1.2mm, 1.54deg; 72x, 0.8mm, 0.69deg; 103x, 0.6mm, 0.77deg \\ 
{\bf Altitude:} & 12m & {\bf Filters:} & Variable Polarising Filter (VPF), Single Polarising Filter (SPF) \\ 
{\bf Temperature:} & 23C (wind: 5-32km/h) & & \\ 
{\bf Seeing:} & 3 - Moderate seeing & & \\ 
{\bf Transparency:} & 5 - Clear & & \\ 
\end{tabular}
% Detailed observation data
\begin{longtable}{ p{0.7in}  p{0.3in}  p{0.6in}  p{0.9in}  p{5.8in} }
\hline 
{\bf Target} & {\bf Cons} & {\bf Type} & {\bf Power} & {\bf Notes} \\ 
\hline 
Sun & Tau & Star & 51x +/- VPF or SPF, 72x +/- VPF, 103x +/- VPF & I enjoyed observing the Sun a lot today. There were more than 30 sunspots and also a hint of granulation when the wind became calm for short moments. Very nice day. The wind did not allow to see Sun granulation most of the time. The seeing improved after 5pm when the wind became more moderate. Many sunspots were visible. Three larger umbrae were surrounded by nice areas of penumbra. One of this had an irregular shape and the South part vanished gradually. This at all powers. It was very attractive. A central area contain 4 well defined sunspots and many little grey spots. Around the larger sunspots, there were brighter and extended areas on the Sun surface. 51x gave the best view most of the time. 72x and 103x were interesting powers but only suitable when the wind was calm, which was rare! A VPF increased the detail noticeably. Apart from reducing image brightness, I appreciate this filter because it stabilises the image, particularly under average seeing. Surprisingly I found that I prefer the view through a SPF rather than VPF. A VPF reduces image brightness, whereas a SPF improves contrast to me. Through a SPF I could see a hint of granulation at 51x even when there was moderate wind. This did not happen with a VPF. I am considering whether separate the two filters. Having them separate would also be quite comfortable when watching planets in the twilight. SPF also improved contrast for all the sunspots. This works as follows: 0 (or 180) degrees shows the brightest image, 90 (or 270) degrees shows the darkest image. For observing both planets and the Sun, I found that I prefer the view when the SPF is positioned at 45 (or 135 or 225 or 315) degrees. The image was still bright, contrast was highest than all other options and the number of details was maximised. Nagler 7mm, Vixen 5mm, and Nagler 3.5mm form my best eyepiece combination for watching the Sun with the TV-60. All of them are really useful, although the first two achieve best results almost every time. The Nagler 3.5mm can show some very nice close-up of umbrae and penumbrae. \\ 
\hline 
\end{longtable} 

% General observation data
\begin{tabular}{ p{0.9in} p{1.3in} p{1.2in} p{5.2in}}
{\bf Date:} & 03/06/2015 & {\bf Telescopes:} & Tele Vue 60 F6 \\ 
{\bf Time:} & 21:40-23:30 & {\bf Eyepieces:} & TV Panoptic 24mm, Nagler 3.5mm, Bresser SA 2x \\ 
{\bf Location:} & Cambridge, UK & {\bf Power, EP, FOV:} & 15x, 4mm, 4.30deg; 103x, 0.6mm, 0.77deg; 206x, 0.3mm, 0.38deg \\ 
{\bf Altitude:} & 12m & {\bf Filters:} & Astronomik OIII, Single Polarising Filter \\ 
{\bf Temperature:} & 12C (wind: 0km/h) & & \\ 
{\bf Seeing:} & 1 - Perfect seeing & & \\ 
{\bf Transparency:} & 5 - Clear & & \\ 
\end{tabular}
% Detailed observation data
\begin{longtable}{ p{0.7in}  p{0.3in}  p{0.6in}  p{0.9in}  p{5.8in} }
\hline 
{\bf Target} & {\bf Cons} & {\bf Type} & {\bf Power} & {\bf Notes} \\ 
\hline 
Jupiter & Cnc & Planet & 103x +/- SPF, 206x + SPF & Just a quick look until the sky became darker. No particular event tonight. It was very nice to see it. The two main belts revealed some subtle detail appearing like tiny shades. In particular these were more detectable in the North Equatorial Belt.   \\ 
Moon & Sgr & Satellite & 103x +/- SPF & Phase 96\%. No many detail were revealed. The moon is not really interesting when full. Craters and seas were detectable but not immersive.  \\ 
M57 & Lyr & Pln Neb & 103x +/- OIII & Ring nebula. The ring was visible with averted vision, but no other detail really. The contrast between the ring and the internal area is much more visible with an OIII filter. Still nice planetary nebula. \\ 
Zeta & Lyr & Dbl star & 15x, 103x & Already split at 15x, but much nicer at 103x.  \\ 
Delta & Lyr & Dbl star & 15x, 103x & This is a very nice system of stars already visible at 15x. A larger triangle with a little internal triangle. Just beautiful. \\ 
Epsilon & Lyr & Dbl star & 15x, 103x, 206x & The Double Double. Epsilon 1 and 2 were easily split at 15x. At 103x it was possible to detect that both Epsilon 1 and 2 are double stars themselves. At 206x this pair of tight double stars was visible although these double stars remained very close. Same colour. \\ 
Beta & Lyr & Dbl star & 15x, 103x & Sheliak. Splendid double already split at 15x. This was very nice at 103x.  \\ 
Beta & Cyg & Dbl star & 15x, 103x & Albireo. Wonderful double. A bit tight at 15x, but very nice at 103x. One orange, the other one blue. \\ 
\hline 
\end{longtable} 

% General observation data
\begin{tabular}{ p{0.9in} p{1.3in} p{1.2in} p{5.2in}}
{\bf Date:} & 26/05/2015 & {\bf Telescopes:} & Tele Vue 60 F6 \\ 
{\bf Time:} & 21:15-0:00 & {\bf Eyepieces:} & TV Panoptic 24mm, Nagler 13mm, Nagler 3.5mm, Bresser SA 2x \\ 
{\bf Location:} & Cambridge, UK & {\bf Power, EP, FOV:} & 15x, 4mm, 4.30deg; 28x, 2.2mm, 2.80deg; 103x, 0.6mm, 0.77deg; 206x, 0.3mm, 0.38deg \\ 
{\bf Altitude:} & 12m & {\bf Filters:} & Astronomik UHC, OIII, Single Polarising Filter \\ 
{\bf Temperature:} & 9-14C (wind: 15km/h) & & \\ 
{\bf Seeing:} & 2 - Slight undulations & & \\ 
{\bf Transparency:} & 5 - Clear & & \\ 
\end{tabular}
% Detailed observation data
\centering 
\begin{longtable}{ p{0.7in}  p{0.3in}  p{0.6in}  p{0.9in}  p{5.8in} }
\hline 
{\bf Target} & {\bf Cons} & {\bf Type} & {\bf Power} & {\bf Notes} \\ 
\hline 
Jupiter & Cnc & Planet & 103x +/- SPF & Observed in civil twilight. The seeing was not enough good for pushing magnification beyond 103x. At 103x, two major belts and two moons were visible. I did not spend much on this target tonight because it was too windy when I observed it. \\ 
Moon & Leo & Satellite & 103x +/- SPF, 206x + SPF & Observed in the twilight. Visible almost 60\% of its phase. The SPF seems to stabilise the image if the seeing is not good. This is a lovely target with the TV60, and keeps magnification pretty well. At 206x the moon surface appeared like a bubble at the poles due to the seeing, but there were moments in which it was possible to see a quasi stable image. Subtle details on the surface were observable as well as minute craters and shades on the ground. Interestingly, on the terminator mounts tips were illuminated whereas their bases were obscured. There is so much to see at 206x that one could spend the entire night observing our satellite! Montes Apenninus, Caucasus, and Alpes were incredible targets and appeared just beautiful. The crater Cassini and all the small nearby craters were spectacular. While I am not sure the SPF increased image contrast, I prefer the view with SPF as it seems that the image is just stabler at both 103x and 106x. \\ 
M57 & Lyr & Pln Neb & 28x +/- UHC, 103x +/- UHC or OIII & Ring Nebula. The UHC filter increases a little bit the visibility of this target at 28x, but does not improve the contrast. The object appears as a grey blob without a shape. At 103x the ring was detectable using an UHC filter using averted vision, but this was not easy too see. The ring shape was more noticeable with a OIII filter despite the severe loss in image brightness. Without filter the nebula appeared just as a grey blob and no ring was detectable. Generally, I think an exit pupil of 0.6mm is just too small for nebula filters. It seems to me that 1.0mm is the maximum usable effectively. As this is the exit pupil typically used when observing planetary nebulae, I would say that an OIII filter is a better choice for these targets as it allows to increase contrast which is needed on these targets. Conversely, for bright extended nebulae to watch with low power eyepieces (or exit pupils larger than 3mm), a UHC filter can be beneficial for targeting and maximizing nebulae extension. \\ 
M97 & UMa & Pln Neb & 28x +/- UHC & Owl Nebula. Invisible at 28x with or without UHC filter. This target requires an OIII filter for being detectable with small aperture telescopes. Consistently with what said for M57, the OIII filter is a better choice for planetary nebulae (and for extended nebulae where we want to maximise nebulae contrast). \\ 
M81 & UMa & Galaxy & 15x, 28x & Bode's nebulae. Not easy to find it at 15x with half moon, but M81 and M82 were detectable via star hopping from Dubhe. At 28x this large galaxy shows its core and a bit of brightness on the body. I was very impressed at seeing these two targets and I believe M31, M32, M101, M81, and M82 are the most appreciable galaxies for small telescopes. Averted vision improved the visibility of this target significantly. \\ 
M82 & UMa & Galaxy & 15x, 28x & Cigar galaxy. Its elongated shape was visible. It was amazing to see this galaxy and its neighbour M81 in the same field. These two targets are going to become one of my favourite objects.   \\ 
M3 & CVn & Glob CL & 28x & As all the globular cluster seen with a small telescope, M3 also appears like a little grey cloud. This is a bright globular cluster and a hint of 'granulation' is perceptible although no star can be resolved. Not very easy to find due to the lack of bright stars to star hop from Arcturus. \\ 
M5 & Ser & Glob CL & 28x & It appears like a grey cloud. From the star Unukalhai (Alpha Ser), go South and you find it. It is a relatively easy target.  \\ 
M13 & Her & Glob CL & 28x & Same as M3. Very bright and large globular cluster. Some granulation is perceptible but no star could be resolved. \\ 
\hline 
\end{longtable} 

% General observation data
\begin{tabular}{ p{1.7in} p{1.2in} p{1.5in} p{4.2in}}
{\bf Date:} & 20/05/2015 & {\bf Telescopes:} & Tele Vue 60 F6 \\ 
{\bf Time:} & 9.30pm-12.00am & {\bf Eyepieces:} & TV Panoptic 24mm, Nagler 13mm, Nagler 3.5mm, Bresser SA 2x \\ 
{\bf Location:} & Cambridge, UK & {\bf Power, Exit pupil, FOV:} & 15x, 4mm, 4.30deg; 28x, 2.2mm, 2.80deg; 103x, 0.6mm, 0.77deg; 206x, 0.3mm, 0.38deg \\ 
{\bf Altitude (MAMSL):} & 12m & {\bf Filters:} & Astronomik OIII, Single Polarising Filter \\ 
{\bf Temperature (C):} & 9 (no wind) & & \\ 
{\bf Seeing (Antoniadi scale):} & 3 - Moderate seeing & & \\ 
{\bf Transparency (AAAA scale):} & 5 - Clear & & \\ 
\end{tabular}
% Detailed observation data
\centering 
\begin{longtable}{ p{0.8in}  p{0.3in}  p{0.5in}  p{0.9in}  p{5.8in} }
\hline 
{\bf Target} & {\bf Cons} & {\bf Type} & {\bf Power} & {\bf Notes} \\ 
\hline 
Jupiter & Cnc & Planet & 103x +/- SPF, 206x + SPF & Observed in the twilight. The SPF noticeably improved the view. Four bands and the transit of Callisto were easily visible at both 103x and 206x. The use of a SPF seemed to stabilise the image and improved contrast. A fair amount of shades were also perceptible on the main two bands. The transit appeared as a crisp black dot on the planet atmosphere. Without the SPF it was only possible to see the two main bands and no shade on them. They simply appeared as two thick lines across the planet. Interestingly the transit shadow appeared a tiny bit better without the filter. To me, using the SPF requires a bit of experience in order to rotate the eyepiece to gain the best contrast. However this is feasible. \\ 
Venus & Gem & Planet & 103x +/- SPF, 206x + SPF & Observed in the twilight. Visible 60\% of its phase. No cloud detail was detectable even with the SPF. Planet glare was reduced but it was very difficult to focus, likely due to the average seeing and high magnification. \\ 
NGC869 / NGC884 - C14 & Per & Opn CL & 15x, 28x & This target is superb with the Nagler 13. Although low on the horizon, it still offers nice contrast with the background sky and the 2.8deg of fov show the object with all its context. \\ 
M56 & Lyr & Glob CL & 15x, 28x & This is the first time I detect this target. It is quite challenging to find with a 60mm but not impossible. I used the Sheliak (Beta Lyrae) and Sulafat (Gamma Lyrae) as pointers to target M56 along the line passing through these two stars on the side of Sulafat. M56 was detectable with a 15x using an atlas but was very faint and averted vision was almost required to find it. At 28x the object was more visible, but did not show much detail as it only appeared as a soft grey patch. This target required larger aperture and / or darker skies. \\ 
M57 & Lyr & Pln Neb & 28x +/- OIII & Ring Nebula. I tried the OIII filter with the Nagler 3.5 (103x). Although the ring shape was noticeable, it was just too much magnification and the overall image was largely degraded. At 28x + OIII the Ring Nebula emerged from the background sky and appeared as a colourless bubble. I believe that an exit pupil of 1-1.5mm can improve the view for this target. \\ 
M97 & UMa & Pln Neb & 28x +/- OIII & Owl Nebula. Interesting target. Completely invisible without OIII filter. With the OIII, it emerges from the sky and the nearby stars. It is a quite large planetary nebula. No colour or shape was detectable, but it simply appeared as a grey bubble. At 15x + OIII was detectable, but was too small to see any major detail. \\ 
M108 & UMa & Galaxy & 15x, 28x & Surfboard Galaxy. Invisible at both 15x and 28x. This object as well as most of the other galaxies require larger aperture and / or a darker sky. \\ 
NGC6992 / NGC6960 - C33 / C34 & Cyg & Glob CL & 15x + OIII & Veil Nebula. Again, no visible although it is too low on the horizon. \\ 
Saturn & Sco & Planet & 103x, 206x & It is still fairly low on the horizon. Titan was visible. The Cassini division was not detectable, but it was possible to see a shade in the middle of the ring. At 206x the image was just degraded and difficult to focus. I have to wait for a higher position of the planet. \\ 
\hline 
\end{longtable} 

% General observation data
\begin{tabular}{ p{1.7in} p{1.2in} p{1.5in} p{4.2in}}
{\bf Date:} & 13/05/2015 & {\bf Telescopes:} & Tele Vue 60 F6 \\ 
{\bf Time:} & 21:00-0:00 & {\bf Eyepieces:} & TV Panoptic 24mm, Nagler 13mm, Nagler 3.5mm, Bresser SA 2x \\ 
{\bf Location:} & Cambridge, UK & {\bf Power, Exit pupil, FOV:} & 15x, 4mm, 4.30deg; 28x, 2.2mm, 2.80deg; 103x, 0.6mm, 0.77deg; 206x, 0.3mm, 0.38deg \\ 
{\bf Altitude (MAMSL):} & 12m & {\bf Filters:} & Astronomik OIII \\ 
{\bf Temperature (C):} & 8 (no wind) & & \\ 
{\bf Seeing (Antoniadi scale):} & 1 - Perfect seeing & & \\ 
{\bf Transparency (AAAA scale):} & 5 - Clear & & \\ 
\end{tabular}
% Detailed observation data
\centering 
\begin{longtable}{ p{0.8in}  p{0.3in}  p{0.5in}  p{0.9in}  p{5.8in} }
\hline 
{\bf Target} & {\bf Cons} & {\bf Type} & {\bf Power} & {\bf Notes} \\ 
\hline 
Jupiter & Cnc & Planet & 103x, 206x & Observed in the twilight. The idea started as a joke because I thought the image would have been too dark for discerning any detail. Instead, it was possible to perceive a little amount of shades on the two major bands of the planet. The boundaries of the other two less visible bands (North and South hemisphere, respectively) were also there. At 103x I was able to see the boundaries of these two bands on the 'equator side', but not on the 'pole side'. At 206x these were noticeable. 4 moons were detectable and one was just about to get closer to Jupiter. I agree with Gerry (sgl: jetstream) that watching Jupiter in twilight shows more contrast. I was also able to see some red-ish colour on the major two bands, which instead is less noticeable when watching Jupiter in the dark. Looking at a bright source before watching the planet did not help me instead. I found I had more difficulty to notice details. Although the exit pupil was only 0.3mm, floaters did not cause me serious problems. Interestingly, I found floaters to be a problem when watching the Sun at 103x. Could these be related to overall image brightness?  \\ 
Venus & Gem & Planet & 103x, 206x & Observed in the twilight. Visible 60\% of its phase. No cloud detail was detectable. At 206x, on one side of the focus, Venus appeared violet, on the other side green/yellow. When in focus, there was no colour aberration. Curiously, at this magnification it was still very bright suggesting that a variable polarising filter might be beneficial.  \\ 
C1 & UMi & Opn CL & 28x & The detail of this target seemed accessible, but I did not manage to see anything in the position suggested by Stellarium. Unfortunately, I did not have a star atlas with me.  \\ 
Alpha & UMi & Dbl Star & 28x, 103x & Polaris. A nice target because of the large difference in brightness between the two stars. Polaris' companion was dim grey and detectable at 28x with some difficulty. 103x easily split the two stars.  \\ 
M60 & Vir & Galaxy & 28x & It was detectable with averted vision at 28x. It appeared as a grey patch without a structure. I tried M58, but it was not visible. I believe that to see these targets with this small telescope, very dark skies are required. \\ 
NGC6992 / NGC6960 - C33 / C34 & Cyg & Glob CL & 15x + OIII & Veil Nebula. No visible, although it is not the best time of the year to see this target. \\ 
M57 & Lyr & Pln Neb & 15x + OIII & Ring Nebula. The OIII filter largely improves the detection of this nebula at 15x. Without a filter, its detection is not easy. It emerges in the sky as a grey little ball. I believe the Nagler 7mm or even the Vixen 5mm can give great views when combined with an OIII filter. \\ 
Saturn & Sco & Planet & 103x & It was a bit higher than yesterday, but unfortunately, my telescope and eyepieces were soaked with humidity and could not really see this target after the first 5 min. Rings were clearly defined, and I believe the Cassini division could have been detectable. \\ 
\hline 
\end{longtable} 

% General observation data
\begin{tabular}{ p{0.9in} p{1.3in} p{1.2in} p{5.2in}}
{\bf Date:} & 12/05/2015 & {\bf Telescopes:} & Tele Vue 60 F6 \\ 
{\bf Time:} & 21:00-23:45 & {\bf Eyepieces:} & TV Panoptic 24mm, Nagler 13mm, Nagler 7mm, Nagler 3.5mm, Bresser SA 2x \\ 
{\bf Location:} & Cambridge, UK & {\bf Power, EP, FOV:} & 15x, 4mm, 4.30deg; 28x, 2.2mm, 2.80deg; 51x, 1.2mm, 1.54deg; 103x, 0.6mm, 0.77deg; 206x, 0.3mm, 0.38deg \\ 
{\bf Altitude:} & 12m & {\bf Filters:} &  \\ 
{\bf Temperature:} & 6C (wind: 20km/h) & & \\ 
{\bf Seeing:} & 2 - Slight undulations & & \\ 
{\bf Transparency:} & 5 - Clear & & \\ 
\end{tabular}
% Detailed observation data
\centering 
\begin{longtable}{ p{0.7in}  p{0.3in}  p{0.6in}  p{0.9in}  p{5.8in} }
\hline 
{\bf Target} & {\bf Cons} & {\bf Type} & {\bf Power} & {\bf Notes} \\ 
\hline 
Jupiter & Cnc & Planet & 103x, 206x & Observed in the twilight. Still visible at 206x with some detail but the new tripod is not up to this sort of magnifications. To be fair, the new tripod was fine at 103x but only when there was no wind. 3 bands and 4 moons visible. It would be interesting to try 206x with my solid tripod. \\ 
Venus & Gem & Planet & 103x & Observed in the twilight. Visible 60\% of its phase. No cloud detail was detectable. A polarised filter might help on this target. \\ 
M44 & Cnc & Opn CL & 15x, 28x & The Nagler 13mm offers the best view. Its fov covers the whole object nicely. Image not degraded at all and the background sky was darkened just the right amount for maximising contrast. Exit pupil of about 2.0mm shows a really nice brightness / contrast for point source DSO. \\ 
NGC869 / NGC884 - C14 & Per & Opn CL & 15x, 28x & Again, the Nagler 13 offering almost 3 degrees of fov shows the full object with great detail but conserving an adequate image brightness.  \\ 
Stock2 & Cas & Opn CL & 15x, 28x & As above.  \\ 
M103 or C13 (NGC457) & Cas & Opn CL & 28x & Not sure whether I saw M103 or C13 (the Owl Cluster) though. A clear double star was well visible and there were a few dim stars in the background were also detectable. This object starts being visible at 28x. It is relatively small, but a lovely target. I think it was M103 as my memory seems more similar to the images. \\ 
M60 & Vir & Galaxy & 15x, 28x & Turn West to Vindemiatrix. A little crown of star is visible slightly South. Continue and you see a little arrow of stars and a single star in the North. M60 is between these two objects. Not detectable at 15x. Detectable via averted vision at 28x. A patch of grey. The sky was not fully dark though and my eye was not dark adapted. I believe this object can show more detail.  \\ 
27 Hya & Hya & Dbl Star & 15x, 28x & Easily split. Colours detectable. \\ 
M13 & Her & Glob CL & 15x, 28x & Detectable at 15x, but nicer at 28x. No star was resolved. It would be interesting to try 51x although I guess this might be too much.  \\ 
M57 & Lyr & Pln Neb & 15x, 28x, 51x & Ring Nebula. For the first time, I managed to see this object with the TV-60. I find extremely difficult to detect it at 15x unless I map the nearby stars with Stellarium. At 28x M57 is clearly visible and appears as a grey blob. At 51x the ring is detectable. I did not try to use an OIII filter because I was freezing due to lack of cloths and about to leave. I believe this target will show much more detail at 51x with OIII filter. \\ 
Saturn & Sco & Planet & 103x & Very low on the horizon and therefore not the best moment for viewing this target. Despite this, rings and titan were visible. Cassini division and bands were not detectable. \\ 
\hline 
\end{longtable} 

% General observation data
\begin{tabular}{ p{0.9in} p{1.3in} p{1.2in} p{5.2in}}
{\bf Date:} & 30/04/2015 & {\bf Telescopes:} & Tele Vue 60 F6 \\ 
{\bf Time:} & 22:00-23:00 & {\bf Eyepieces:} & TV Panoptic 24mm, Nagler 7mm, Nagler 3.5mm \\ 
{\bf Location:} & Cambridge, UK & {\bf Power, EP, FOV:} & 15x, 4mm, 4.30deg; 51x, 1.2mm, 1.54deg; 103x, 0.6mm, 0.77deg \\ 
{\bf Altitude:} & 12m & {\bf Filters:} &  \\ 
{\bf Temperature:} & 6C (no wind) & & \\ 
{\bf Seeing:} & 2 - Slight undulations & & \\ 
{\bf Transparency:} & 5 - Clear & & \\ 
\end{tabular}
% Detailed observation data
\centering 
\begin{longtable}{ p{0.7in}  p{0.3in}  p{0.6in}  p{0.9in}  p{5.8in} }
\hline 
{\bf Target} & {\bf Cons} & {\bf Type} & {\bf Power} & {\bf Notes} \\ 
\hline 
Moon & Vir & Satellite & 51x, 103x & Moon phase 91\%. The moon at 103x did not need a filter. It was very crisp and showed details in the south hemisphere despite it was almost full. At 51x, the moon is simply scaled of a factor of 1/2, indicating that the Nagler 3.5mm behaves as a perfect 2x Nagler 7mm. It would be useful to have a Moon map to check the crater's names.  \\ 
Jupiter & Cnc & Planet & 103x & At 103x Jupiter showed 4 moons and 4 bands. No specific events were visible this evening. \\ 
M44 & Cnc & Opn CL & 15x & Beehive cluster. One of the best wide open cluster. Not many faint stars were visible because of almost full moon. However, the cluster still emerged in the sky. \\ 
NGC869 / NGC884 - C14 & Per & Opn CL & 15x, 51x & Double cluster. Superb cluster. Even if low in the sky, it was a pleasure to see it. At 51x, the cluster emerged from the background and showed more faint stars. \\ 
Stock2 & Cas & Opn CL & 15x & This cluster next to the double cluster is very nice and needs a wide field telescope. Its stars are not so bright and generally of similar magnitude and colour. \\ 
Mel20 & Per & Opn CL & 15x & Alpha Per moving cluster. Wide field telescopes or binoculars are the best for this superb cluster. Even if low in the sky, it was very beautiful to see. >70 stars detectable. \\ 
NGC2392 - C39 & Gem & Pln Neb & 15x, 51x & Eskimo nebula. At 15x it was detectable with averted vision. It was easily visible at 51x and appeared like a fuzzy blue/grey small patch next to the star.  \\ 
\hline 
\end{longtable} 

% General observation data
\begin{tabular}{ p{1.7in} p{1.2in} p{1.5in} p{4.2in}}
{\bf Date:} & 14/04/2015b & {\bf Telescopes:} & Tele Vue 60 F6 \\ 
{\bf Time:} & 9.30pm-11.20pm & {\bf Eyepieces:} & TV Panoptic 24mm, Plossl 20mm, Nagler 7mm T6, Vixen SLV 5mm, Nagler 3.5mm T6 \\ 
{\bf Location:} & Cambridge, UK & {\bf Power, Exit pupil, FOV:} & 15x, 4mm, 4.30deg; 18x, 3.3mm, 2.7deg; 51x, 1.2mm, 1.54deg; 72x, 0.8mm, 0.69deg; 103x, 0.6mm, 0.77deg \\ 
{\bf Altitude (MAMSL):} & 12m & {\bf Filters:} &  \\ 
{\bf Temperature (C):} & 9 (no wind) & & \\ 
{\bf Seeing (Antoniadi scale):} & 2 - Slight undulations & & \\ 
{\bf Transparency (AAAA scale):} & 5 - Clear & & \\ 
\end{tabular}
% Detailed observation data
\centering 
\begin{longtable}{ p{0.8in}  p{0.3in}  p{0.5in}  p{0.9in}  p{5.8in} }
\hline 
{\bf Target} & {\bf Cons} & {\bf Type} & {\bf Power} & {\bf Notes} \\ 
\hline 
M35 & Gem & Opn CL & 15x & Under dark sky this object emerges clearly. A few bright stars with many little faint stars in background. Averted vision helps, but this object is not too demanding if the sky is sufficiently transparent. \\ 
M65 & Leo & Galaxy & 15x, 18x & This object requires aperture and dark sky to be detected and viewed properly. Just very faint object visible through averted vision. An exit pupil of 3.3mm is better than 4.0mm. I wonder whether something between 2.5 and 2.0mm can improve this view even more. \\ 
M66 & Leo & Galaxy & 15x, 18x & See above \\ 
Mel111 & Com & Opn CL & 15x & Coma Berenices star cluster. Located just south of Gamma Com, this large object is as spectacular as M44. Very rich of stars, some bright some small and fainting. Some are doubles. As this is a large object, 15x is the adequate magnification. As Coma Berenices does not have bright stars, you can find this object knowing that is between Denebola (Leo) and Cor Caroli (Alpha CVn). \\ 
M53 & Com & Glob CL & 15x, 18x, 51x & This object is not easy to find. It is just 1-2 degrees east of Diadem (Alpha Com), but this star is very dim to be seen, unless the sky is enough dark. Instead use the Virgo trapezium and point to north following the star Vindemiatrix (Virgo). You can find Diadem just going some degree north from Vindemiatrix. M53 appears as a little grey cloud at 15x. No detail of this globular cluster is visible. At 18x, the contrast is a bit improved, but the image is the same. At 51x this objects  is larger and well detectable, but still appears like a grey cloud. \\ 
NGC5053 & Com & Glob CL & 15x, 18x, 51x & Invisible. This is a bit smaller and dimmer than M53. I could not find it. \\ 
M3 & CVn & Glob CL & 15x, 51x & Again, not easy to find. I used the axis from Gamma to Beta Com. This cluster is brighter than M54 and at 51x seems a large white/grey blob.  \\ 
Jupiter & Cnc & Planet & 103x, 72x & Transit of Ganymede on Jupiter. Little black dot on the Equatorial zone. All the other three main satellites were well distict on right. \\ 
\hline 
\end{longtable} 

\begin{tabular}{ p{1.7in} p{1.2in} p{1.5in} p{4.2in}}
{\bf Date:} & 14/04/2015a & {\bf Telescopes:} & Tele Vue 60 F6 \\ 
{\bf Time:} & 6.00pm-7.00pm & {\bf Eyepieces:} & Nagler 7mm, Vixen SLV 5mm, Nagler 3.5mm \\ 
{\bf Location:} & Cambridge, UK & {\bf Power, Exit pupil, FOV:} & 51x, 1.2mm, 1.54deg; 72x, 0.8mm, 0.69deg; 103x, 0.6mm, 0.77deg \\ 
{\bf Altitude (MAMSL):} & 12m & {\bf Filters:} & Variable Polarizing Filter \\ 
{\bf Temperature (C):} & 21 (no wind) & & \\ 
{\bf Seeing (Antoniadi scale):} & 2 - Slight undulations & & \\ 
{\bf Transparency (AAAA scale):} & 5 - Clear & & \\ 
\end{tabular}
% Detailed observation data
\centering 
\begin{longtable}{ p{0.8in}  p{0.3in}  p{0.5in}  p{0.9in}  p{5.8in} }
\hline 
{\bf Target} & {\bf Cons} & {\bf Type} & {\bf Power} & {\bf Notes} \\ 
\hline 
Sun & Psc & Star & 51x +/- VPF; 72x, 103x & Today at 4pm there was a gigantic flare (CME) about 1 sun radius long. Unfortunately I was not at home. I looked at the Sun, but the flare was gone by the time I set up the telescope. A large group of black spots was visible in the North hemisphere. Around them granulation was clearly visible. Granulation was also detectable, although with some difficulty, on the Sun surface at 51x using a VPF. At 72x the Sun revealed a nice image where Sun spot details were visible as well as surface granulation. 103x was just too much for this seeing. Although it can be used for magnifying the solar spots, granulation is completely lost. In addition, floaters become a real issue when watching the sun using 0.6mm exit pupil. I think the best magnification is between 51x and 72x. The Vixen 5mm works very well with the Sun. This was used without VPF filter. \\ 
\hline 
\end{longtable} 

% General observation data
\begin{tabular}{ p{1.7in} p{1.2in} p{1.5in} p{4.2in}}
{\bf Date:} & 11/04/2015 & {\bf Telescopes:} & Tele Vue 60 F6 \\ 
{\bf Time:} & 6.00pm-7.00pm & {\bf Eyepieces:} & Nagler 7mm, Vixen SLV 5mm \\ 
{\bf Location:} & Cambridge, UK & {\bf Power, Exit pupil, FOV:} & 51x, 1.2mm, 1.54deg; 72x, 0.8mm, 0.69deg \\ 
{\bf Altitude (MAMSL):} & 12m & {\bf Filters:} & Variable Polarizing Filter \\ 
{\bf Temperature (C):} & 12 (wind) & & \\ 
{\bf Seeing (Antoniadi scale):} & 2 - Slight undulations & & \\ 
{\bf Transparency (AAAA scale):} & 2 - Poor & & \\ 
\end{tabular}
% Detailed observation data
\centering 
\begin{longtable}{ p{0.8in}  p{0.3in}  p{0.5in}  p{0.9in}  p{5.8in} }
\hline 
{\bf Target} & {\bf Cons} & {\bf Type} & {\bf Power} & {\bf Notes} \\ 
\hline 
Sun & Psc & Star & 51x +/- VPF; 72x & Two spot areas: one in the north, the other one in the south. 5 small spots where visible in the north, 4-5 in the south spot area. 2 small spots in the centre of the sun. Best view 51x, using variable polarising filter. No granularity was visible due to the bad seeing. \\ 
\hline 
\end{longtable} 

% General observation data
\begin{tabular}{ p{0.9in} p{1.3in} p{1.2in} p{5.2in}}
{\bf Date:} & 09/04/2015 & {\bf Telescopes:} & Tele Vue 60 F6 \\ 
{\bf Time:} & 21:20-22:45 & {\bf Eyepieces:} & TV Panoptic 24mm, Nagler 3.5mm \\ 
{\bf Location:} & Cambridge, UK & {\bf Power, EP, FOV:} & 15x, 4mm, 4.30deg; 103x, 0.6mm, 0.77deg \\ 
{\bf Altitude:} & 12m & {\bf Filters:} & Variable Polarizing Filter \\ 
{\bf Temperature:} & 9C (no wind) & & \\ 
{\bf Seeing:} & 2 - Slight undulations & & \\ 
{\bf Transparency:} & 2 - Poor & & \\ 
\end{tabular}
% Detailed observation data
\centering 
\begin{longtable}{ p{0.7in}  p{0.3in}  p{0.6in}  p{0.9in}  p{5.8in} }
\hline 
{\bf Target} & {\bf Cons} & {\bf Type} & {\bf Power} & {\bf Notes} \\ 
\hline 
Jupiter & Cnc & Planet & 15x; 103x +/- VPF & At the eyepiece from right to left: Callisto, Io, Jupiter, Europa and Ganymede. This evening I decided to test my new eyepiece (Nagler 3.5mm). Due to the lack of transparency, I only tested this on Jupiter. This was the first time I observed at 103x without using a Bresser 2x SA. The difference was quite substantial. I had the impression that the Nagler 7mm with Bresser 2x SA was more colour corrected than the Nagler 3.5mm only at the edge (last 10\% before the field stop). This might have been caused by the presence of light fog though, instead of the eyepiece. I will test this again. On the other hand, the lack of the Bresser 2x SA (4 lens less) improved transparency, and this was detectable. With a Nagler 7mm and Bresser 2x SA, I am able to see a bit more than the two main bands only when the seeing is quite good. Tonight, although the seeing was acceptable, but the sky was quite foggy. The main two bands (North and South Equatorial Belts) were visible and other two bands at the poles were easily detectable (North Polar Region, S.S. Temperate Belt). In the North and South Equatorial Belts, some shades were also detectable. No direction was visible but it was possible to see that the borders and belt colours were rough and not homogeneous. This was particularly true for the North Equatorial Band. No GRS was detectable. The use of a single or double polarizing filter did not improve image quality. The whole image only appeared too dark and the minute details previously described were lost. Possibly, the VPF is more appropriate for brighter objects (e.g. the Sun and the Moon) or Jupiter during sunset or dawn.  \\ 
\hline 
\end{longtable} 

% General observation data
\begin{tabular}{ p{1.7in} p{1.2in} p{1.5in} p{4.2in}}
{\bf Date:} & 06/04/2015 & {\bf Telescopes:} & Tele Vue 60 F6 \\ 
{\bf Time:} & 9pm-10.45pm & {\bf Eyepieces:} & TV Panoptic 24mm, Nagler 7mm, Vixen SLV 5mm, Bresser 2x SA \\ 
{\bf Location:} & Cambridge, UK & {\bf Power, Exit pupil, FOV:} & 15x, 4mm, 4.30deg; 30x, 2mm, 2.15deg; 51x, 1.2mm, 1.54deg; 72x, 0.8mm, 0.69deg; 103x, 0.6mm, 0.77deg \\ 
{\bf Altitude (MAMSL):} & 12m & {\bf Filters:} & Astronomik UHC, OIII \\ 
{\bf Temperature (C):} & 8 (no wind) & & \\ 
{\bf Seeing (Antoniadi scale):} & 2 - Slight undulations & & \\ 
{\bf Transparency (AAAA scale):} & 3 - Somewhat clear & & \\ 
\end{tabular}
% Detailed observation data
\centering 
\begin{longtable}{ p{0.8in}  p{0.3in}  p{0.5in}  p{0.9in}  p{5.8in} }
\hline 
{\bf Target} & {\bf Cons} & {\bf Type} & {\bf Power} & {\bf Notes} \\ 
\hline 
NGC1647 & Tau & Opn CL & 15x, 51x & In this period of the year, Taurus is quite low on the horizon. This object did not show many stars due to the light pollution and atmosphere. You will need darker skies or look at it when it is higher on the horizon. From Aldebaran, go east for 4 degrees. It is near a couple of stars and i Tauri (mag 5), which is the brightest star in the area. 51x did not help tonight, as the object became too dim. I suspect a ~30x, exit pupil 2.0mm would be the best for these targets.  \\ 
NGC1746 & Tau & Opn CL & 15x, 51x & After finding NGC1647, from i Tauri, go east for other 4-5deg until you see iota(?) Tauri (102 Tau, mag 4.6). NGC1746 is a medium size cluster at 15x at west of 102 Tau. It also appears very nice at 51x. \\ 
NGC1817 & Tau & Opn CL & 15x, 51x & After finding NGC1647, from i Tauri, go south-east for 5 deg until you see m Tauri (mag 4.9). NGC1817 and 1807 appear on the same field of view at both 15x and 51x. 15x is not sufficient for seeing details of these two clusters. These are accessible at 51x. NGC1817 seems less visible than NGC1807 \\ 
NGC1807 & Tau & Opn CL & 15x, 51x & See NGC1817. NGC1807 is more spectacular than NGC1817 at 51x possibly because it has brighter stars.  \\ 
NGC1662 & Ori & Opn CL & 15x, 51x & A little open cluster near the top of Orion's shield. It is also reachable from Aldebaran (Alpha Tauris) moving towards south for about 8deg. At 15x it only shows it compactness. At 51x it shows some stars. This is a compact open cluster formed by relatively visible stars.   \\ 
M1 & Tau & SN Rem & 15x; 51x +/- OIII, UHC & Invisible after trying with different magnifications and averted vision. Filters did not help either. You need dark skies for this target. Possibly you need to use an eyepiece at 2.0mm exit pupil. \\ 
M65 & Leo & Galaxy & 30x & Leo triplet. Elongated grey patch visible with averted vision. Shape of a cigar. At 30x, the patch is visible more easily than at 15x. The leo triplet is more easily detectable when the telescope is slightly moved. The patches will move accordingly. \\ 
M66 & Leo & Galaxy & 30x & Leo triplet. As for M65.  \\ 
M51 & CVn & Galaxy & 30x & Whirlpool Galaxy. Grey patch without a structure visible with averted vision. A darker sky will make the difference on this target. \\ 
Jupiter & Cnc & Planet & 103x & Two bands clearly visible and a faint one on the south hemisphere was detectable. All four satellite were visible. Io and Europa were very tight at East of the planet. \\ 
NGC3242 - C59 & Hya & Pln Neb & 15x; 51x +/- OIII, UHC; 72x & Ghost of Jupiter. By naked eye, from Alphard (Alpha Hydrae, mag 1.95), move east and detect the Lambda Hydrae (mag 3.6). This star appears like a star system extending north and south from Lambda Hydrae. Continue moving east following Hydrae body. The next star is slightly south of Lambda. This is Mu Hydra (mag 3.6). Then next one is Nu Hydra (mag 3.10). Mu Hydra will appear Yellow/Orange and almost isolated. It has a little star on the north. Slightly south, you see two bright couples of stars: two more distant at east (HIP50693, HIP50764), two closer at west (HIP51170, HIP51193). Consider the tight couple at west. There is a little star (near this couple in the direction of the other couple. If you use the tight couple and the little star as pointer and you move for another segment in the direction of the little star, the planetary nebula will appear. This appears as a faint tiny and diffuse light. No structure. At 51x it appears like a little full circle. An OIII seems more effective than an UHC filter here possibly because the planetary nebula is low on the horizon. The OIII filter makes it appear from the sky, whereas really few nearby stars are visible. 72x does not show more detail. UHC filter works fine but does not boost the object at the same level as the OIII does. \\ 
\hline 
\end{longtable} 

% General observation data
\begin{tabular}{ p{0.9in} p{1.3in} p{1.2in} p{5.2in}}
{\bf Date:} & 25/03/2015 & {\bf Telescopes:} & Tele Vue 60 F6 \\ 
{\bf Time:} & 21:00-22:45 & {\bf Eyepieces:} & TV Panoptic 24mm, Plossl 20mm, Nagler 7mm, Vixen SLV 5mm, Bresser 2x SA \\ 
{\bf Location:} & Cambridge, UK & {\bf Power, EP, FOV:} & 15x, 4mm, 4.30deg; 18x, 3.3mm, 2.73deg; 30x, 2mm, 2.15deg; 51x, 1.2mm, 1.54deg; 72x, 0.8mm, 0.69deg \\ 
{\bf Altitude:} & 12m & {\bf Filters:} & Astronomik UHC, OIII \\ 
{\bf Temperature:} & 6C (no wind) & & \\ 
{\bf Seeing:} & 2 - Slight undulations & & \\ 
{\bf Transparency:} & 3 - Somewhat clear & & \\ 
\end{tabular}
% Detailed observation data
\centering 
\begin{longtable}{ p{0.7in}  p{0.3in}  p{0.6in}  p{0.9in}  p{5.8in} }
\hline 
{\bf Target} & {\bf Cons} & {\bf Type} & {\bf Power} & {\bf Notes} \\ 
\hline 
M47 & Pup & Opn CL & 15x & Rich of stars. These are quite spread, making this cluster easy to detect and study. \\ 
M46 & Pup & Opn CL & 15x & This is a compact cluster. It is detectable.  \\ 
M48 & Hya & Opn CL & 15x & Dim open cluster. It requires transparent skies to shine properly. \\ 
M65 & Leo & Galaxy & 15x, 18x, 30x & Invisible. Sky not transparent enough. I think an exit pupil of 3.3mm is a good compromise between 4mm and 2mm. 2mm is too much for the TV60 on this targets. \\ 
M66 & Leo & Galaxy & 15x, 18x, 30x & Invisible. Sky not transparent enough. \\ 
NGC2244 & Mon & Opn CL & 15x & Satellite cluster. Six stars in two columns  \\ 
NGC2264 & Mon & CL+Neb & 15x & Christmas tree.  \\ 
NGC2392 & Gem & Pln Neb & 15x, 51x +/- OIII, UHC, 72x & Eskimo nebula, C39. From Wasat (Delta Gem) move east to 63 Gem. 63 Gem is the brightest star of a 'half moon' of 7 stars. The Eskimo nebula is next to the star HIP36370 (mag8.2), which is a bit isolated but very close to 63 on the opposite direction of Wasat. You can spot it at 15x without filters, but you see it only with averted vision. It appears as a very small patch next to the star. At 51x the nebula is visible as a grey little ball. The boundaries are obfuscated. An UHC filter helps increasing the contrast between the sky and the nebula. An OIII filter shows even more contrast, although I think an UHC filter is better at this exit pupil (1.2mm). Using these filters, the boundaries of the nebula appear much clearer although no structure is visible at this magnification. At 72x (and no filter) is still visible as a grey little ball. Boundaries are obfuscated.   \\ 
Jupiter & Cnc & Planet & 72x & Quick observation. Two bands and four satellite were visible.  \\ 
Alpha & Hya & Star & 72x & Alphard. Yellow star \\ 
Alpha & Vir & Star & 72x & Spica. Blue star \\ 
Alpha & Leo & Dbl Star & 15x, 72x & Regulus. Blue-white double star visible at 15x. Clearly split at 72x although not all this magnification is actually required for split it. \\ 
\hline 
\end{longtable} 

% General observation data
\begin{tabular}{ p{1.7in} p{1.2in} p{1.5in} p{4.2in}}
{\bf Date:} & 24/03/2015 & {\bf Telescopes:} & Tele Vue 60 F6 \\ 
{\bf Time:} & 7pm-9.30pm & {\bf Eyepieces:} & TV Panoptic 24mm, Nagler 7mm, Vixen SLV 5mm, Bresser 2x SA \\ 
{\bf Location:} & Cambridge, UK & {\bf Power, Exit pupil, FOV:} & 15x, 4mm, 4.30deg; 51x, 1.2mm, 1.54deg; 72x, 0.8, 0.69deg; 144x, 0.4mm, 0.35deg \\ 
{\bf Altitude (MAMSL):} & 12m & {\bf Filters:} &  \\ 
{\bf Temperature (C):} & 5 (no wind) & & \\ 
{\bf Seeing (Antoniadi scale):} & 3 - Moderate seeing & & \\ 
{\bf Transparency (AAAA scale):} & 3 - Somewhat clear & & \\ 
\end{tabular}
% Detailed observation data
\centering 
\begin{longtable}{ p{0.8in}  p{0.3in}  p{0.5in}  p{0.9in}  p{5.8in} }
\hline 
{\bf Target} & {\bf Cons} & {\bf Type} & {\bf Power} & {\bf Notes} \\ 
\hline 
M47 & Pup & Opn CL & 15x, 51x & Superb anchor shape. 51x reveals many more details. Quite tight double star near the centre. From Alpha Mon, go 5 degrees South. \\ 
M46 & Pup & Opn CL & 15x, 51x & Missed as I confused it with the aggregation of stars at North-West of 2 Pup A and 4 Pup.  \\ 
M48 & Hya & Opn CL & 15x & Not easy to see if the sky is not very dark. Fortunately it is quite large. It is the third vertex of the triangle formed by the trio of stars "1Hya C Hya 2Hya (CHya mag 3.8 is the brightest in the middle of the trio) and 29 Mon (mag 4.35).  \\ 
Moon & Tau & Satellite & 72x & Waxing crescent 25\%. Very crisp details.  \\ 
Venus & Cet & Planet & 72x, 144x & 144x was too much. Possibly because the planet was too low, or because the eyepiece did not cool down properly. In any case, even at 72x, no planet atmosphere detail.  \\ 
Mintaka & Ori & Dbl Star & 72x & Easily split.  \\ 
Alnitak & Ori & Dbl Star & 72x & Separated components A and C. Cannot remember if B was detectable. \\ 
Sigma & Ori & Dbl Star & 72x & Visible 5 stars, although only able to split Sigma Orionis into two stars. \\ 
Hatsya & Ori & Dbl Star & 72x & Very nice double star. The companion is quite dim and small compared to Hatsya. Companion is grey.  \\ 
Sirius & CMa & Dbl Star & 72x & Not able to split Sirius. Too much bright. \\ 
Castor & Gem & Dbl Star & 72x & The companion Castor B is also bright and the two stars are quite close to each other. Same colour. Castor C is very dim and more distant from the other two. \\ 
Mekbuda & Gem & Dbl Star & 72x & Easily split double star. \\ 
M35 & Gem & Opn CL & 72x & Many other stars are visible.  \\ 
\hline 
\end{longtable} 

\begin{tabular}{ p{1.7in} p{1.2in} p{1.5in} p{4.2in}}
{\bf Date:} & 22/03/2015 & {\bf Telescopes:} & Tele Vue 60 F6 \\ 
{\bf Time:} & 7pm-10pm & {\bf Eyepieces:} & TV Panoptic 24mm, Nagler 7mm, Vixen SLV 5mm, Bresser 2x SA \\ 
{\bf Location:} & Cambridge, UK & {\bf Power, Exit pupil, FOV:} & 15x, 4mm, 4.30deg; 51x, 1.2mm, 1.54deg; 144x, 0.4mm, 0.35deg \\ 
{\bf Altitude (MAMSL):} & 12m & {\bf Filters:} & Astronomik OIII \\ 
{\bf Temperature (C):} & 3 (no wind) & & \\ 
{\bf Seeing (Antoniadi scale):} & 2 - Slight undulations & & \\ 
{\bf Transparency (AAAA scale):} & 3 - Somewhat clear & & \\ 
\end{tabular}
% Detailed observation data
\centering 
\begin{longtable}{ p{0.8in}  p{0.3in}  p{0.5in}  p{0.9in}  p{5.8in} }
\hline 
{\bf Target} & {\bf Cons} & {\bf Type} & {\bf Power} & {\bf Notes} \\ 
\hline 
M45 & Tau & Opn CL & 15x, 51x & Very clear and defined. 15x offers the best fov. \\ 
M42 & Ori & CL+Neb & 15x + OIII, 51x & 4mm exit pupil + OIII shows nebula extension. 51x shows trapezium \\ 
Sigma & Ori & Mlt star & 51x & Sufficient for seeing 5 stars \\ 
NGC1980 & Ori & Neb & 15x + OIII & 4mm exit pupil + OIII shows a bit of nebula around the star Hatsya \\ 
M78 & Ori & Neb & 15x & Unsuccess \\ 
NGC2244 & Mon & Opn CL & 15x & Satellite cluster. Six stars in two columns  \\ 
NGC2237 & Mon & Neb & 15x + OIII & Rosette nebula. Detectable with OIII filter. A grey patch 2 degree large. No structure visible \\ 
NGC2264 & Mon & CL+Neb & 15x + OIII & Christmas tree + Cone nebula. Christmas tree is easily visible. Cone nebula is detectable with an OIII filter near and south of 15mon. \\ 
M35 & Gem & Opn CL & 15x & Under transparent night, many stars are visible inside. \\ 
M36 & Aur & Opn CL & 15x & Easy to find after finding M38. A bit difficult to see inside as it is quite dim. \\ 
M37 & Aur & Opn CL & 15x & Easy to find after finding M36. Still difficult to see inside. \\ 
M38 & Aur & Opn CL & 15x & Quite clear under transparent skies. \\ 
M44 & Cnc & Opn CL & 15x & Praesepe. Spectacular at 15x. \\ 
M67 & Cnc & Opn CL & 15x, 51x & King cobra. Not to easy to detect. Nicer at 51x. \\ 
M65 & Leo & Galaxy & 15x & Leo triplet. From Chertan (theta Leo), use the star pointers HIP54688 and HIP54711 to reach eta Leo. Eta Leo forms a 90Deg triangle with HIP55033 and HIP55262. From the latter look at south slightly. Galaxy detectable as patches. M56 is elongated. Averted vision for 10min is required. Cover the other eye to relax the observing eye nerve. \\ 
M66 & Leo & Galaxy & 15x & Leo triplet. As for M65. Maybe using an exit pupil of 2.7-2.0mm is better. \\ 
M95 & Leo & Galaxy & 15x & Unsuccess \\ 
M96 & Leo & Galaxy & 15x & Unsuccess \\ 
Pherkad-Pherkad minor & UMi & Dbl Star & 15x & Gamma Ursa Minor blue 3mag. It has a neighbour star 10.30mag. Pherkad Minor orange 5mag. \\ 
19-20 & Dra & Dbl Star & 15x & 4.5mag and 7mag. \\ 
Eta-HIP80309A & Dra & Dbl Star & 15x & 2.7mag and 6.05mag \\ 
M51 & CVn & Galaxy & 15x & Whirlpool Galaxy. From UMA-Alkaid, move south to 24CVn. Continue on that direction until HIP65768. This forms a triangle with HIP66004 and HIP66116. They are all 7mag stars. HIP65768 is the brightest in the area. M51 lies externally of the line between HIP65768 and HIP66004. Averted vision for 10min is required. You will see a grey patch. No structure. \\ 
M101 & UMa & Galaxy & 15x & Unsuccess \\ 
Jupiter & Cnc & Planet & 144x & Order: Europa, Callisto, Jupiter, Io, Ganymede. Two bands very visible. The lower one was visible on the left (refractor). On the right the great red spot was detectable. Very minor bands north and south.  \\ 
\hline 
\end{longtable} 

% General observation data
\begin{tabular}{ p{0.9in} p{1.3in} p{1.2in} p{5.2in}}
{\bf Date:} & 23/02/2015 & {\bf Telescopes:} & Tele Vue 60 F6 \\ 
{\bf Time:} & 19:00-21:00 & {\bf Eyepieces:} & TV Panoptic 24mm, Plossl 20mm, Nagler 7mm, Vixen SLV 5mm, Bresser 2x SA \\ 
{\bf Location:} & Cambridge, UK & {\bf Power, EP, FOV:} & 15x, 4mm, 4.30deg; 18x, 3.3mm, 2.73deg; 103x, 0.6mm, 0.77deg; 144x, 0.4mm, 0.35deg \\ 
{\bf Altitude:} & 12m & {\bf Filters:} & Astronomik OIII, UHC \\ 
{\bf Temperature:} & 1C (no wind) & & \\ 
{\bf Seeing:} & 2 - Slight undulations & & \\ 
{\bf Transparency:} & 3 - Somewhat clear & & \\ 
\end{tabular}
% Detailed observation data
\centering 
\begin{longtable}{ p{0.7in}  p{0.3in}  p{0.6in}  p{0.9in}  p{5.8in} }
\hline 
{\bf Target} & {\bf Cons} & {\bf Type} & {\bf Power} & {\bf Notes} \\ 
\hline 
M42 & Ori & CL+Neb & 15x, 18x & 4mm exit pupil + UHC shows nebula extension. OIII shows more contrast. \\ 
NGC2244 & Mon & Opn CL & 15x, 18x & Satellite cluster. Six stars in two columns  \\ 
NGC2237 & Mon & Neb & 15x, 18x & Rosette nebula. Detectable with OIII filter. Very soft grey patch. No structure. Invisible with UHC filter. \\ 
M35 & Gem & Opn CL & 15x & Detectable but not much detailed. \\ 
M36 & Aur & Opn CL & 15x & Very poor detail, but detectable.  \\ 
M37 & Aur & Opn CL & 15x & Very poor detail, but detectable.  \\ 
M38 & Aur & Opn CL & 15x & The first of the group to find. Poor detail. \\ 
M44 & Cnc & Opn CL & 15x & Praesepe. Spectactular at 15x. \\ 
M67 & Cnc & Opn CL & 15x & King cobra. Not to easy to detect. Looks like a grey patch, more visible using averted vision. \\ 
Jupiter & Cnc & Planet & 103x, 144x & A bit of wind, but the image stays crisp at high magnifications. No aberration. \\ 
NGC1662 & Ori & Opn CL & 15x, 51x & Found casually while scanning from Aldebaran to Beltegeuse. Very small little open cluster at 15x. Much better at 51x. Not easily detectable, because of its small size \\ 
NGC1647 & Tau & Opn CL & 15x  & Beautiful open cluster easily detectable from Aldebaran \\ 
NGC1746 & Tau & Opn CL & 15x & Not sure I found it. It appeared smaller than NGC1647. Possibly a darker sky reveals more interesting features. \\ 
Cr65 & Tau & Opn CL & 15x & Nice aggregation of stars, although none of them really emerges. Wide field is required. North of Orion-Meissa. \\ 
Cr70 & Ori & Opn CL & 15x & Gorgeous Orion's belt. The chain of stars surrounding Alnilam is superb. Wide field telescope. \\ 
\hline 
\end{longtable} 

% General observation data
\begin{tabular}{ p{0.9in} p{1.3in} p{1.2in} p{5.2in}}
{\bf Date:} & xx/xx/1998 & {\bf Telescopes:} & Celestron Newton 114mm F8; Binoculars 15x70; Tele Vue 60 F6 \\ 
{\bf Time:} & Mar 1998 to Jan 2015 & {\bf Eyepieces:} & Kellner 25mm, Orion Sirius 10mm, Orion Shorty 2x Barlow. TV Panoptic 24mm, Plossl 20mm, Nagler 7mm, Vixen SLV 5mm, Bresser 2x SA \\ 
{\bf Location:} & Venice, Lorenzago (IT). Newcastle, Luton Devon, Cambridge, (UK) & {\bf Power, EP, FOV:} & 36.4x, 3.1mm, 1.4deg; 72.8x, 1.6mm, 0.7deg; 91x, 1.3mm, 0.57deg; 182x, 0.6mm, 0.3deg. 15x, 4mm, 4.30deg; 18x, 3.3mm, 2.73deg; 103x, 0.6mm, 0.77deg; 144x, 0.4mm, 0.35deg \\ 
{\bf Altitude:} & 10m - 880m & {\bf Filters:} &  \\ 
{\bf Temperature:} & -5 to +30C (wind: 0 to ~40km/h) & & \\ 
{\bf Seeing:} & 1 to 3 & & \\ 
{\bf Transparency:} & 3 to 5 & & \\ 
\end{tabular}
% Detailed observation data
\centering 
\begin{longtable}{ p{0.7in}  p{0.3in}  p{0.6in}  p{0.9in}  p{5.8in} }
\hline 
{\bf Target} & {\bf Cons} & {\bf Type} & {\bf Power} & {\bf Notes} \\ 
\hline 
M1 & Tau & SN Rem & 36x & Crab Nebula. C114F8, Venice (IT). \\ 
M2 & Aqr & Glob CL & 15x & B15x70, Newcastle (UK). \\ 
M7 & Sco & Opn CL & 36x & Ptolemy Cluster. C114F8, Venice (IT). Large and beautiful open cluster. It was just above the horizon and the last cluster visible if moving towards East. \\ 
M10 & Oph & Glob CL & 36x; 15x & C114F8, Venice (IT); B15x70, Newcastle (UK). \\ 
M11 & Sct & Opn CL & 15x & Wild duck cluster. B15x70, Newcastle (UK). \\ 
M12 & Oph & Glob CL & 15x & B15x70, Newcastle (UK). \\ 
M13 & Her & Glob CL & 36x; 15x, 51x & Hercules Globular Cluster. C114F8, Venice (IT); B15x70, TV60F6, Newcastle (UK). \\ 
M15 & Peg & Glob CL & 15x & B15x70, Newcastle (UK). \\ 
M20 & Sgr & CL+Neb & 36x & Trifid nebula. C114F8, Venice (IT). \\ 
M27 & Vul & Pln Neb & 15x & Dumbbell nebula. B15x70, Newcastle (UK). \\ 
M29 & Cyg & Opn CL & 15x & B15x70, Newcastle (UK). \\ 
M31 & And & Galaxy & 36x; 15x & Andromeda Galaxy. C114F8, Venice (IT); B15x70, Luton, Devon (UK); TV60F6, Newcastle (UK). \\ 
M32 & And & Galaxy & 15x & Companion to M31. B15x70, Luton, Devon (UK). \\ 
M34 & Per & Opn CL & 15x & B15x70, TV60F6, Newcastle (UK). \\ 
M35 & Gem & Opn CL & 15x, 72x & B15x70, Newcastle (UK). \\ 
M36 & Aur & Opn CL & 15x, 30x & TV60F6, Newcastle, Cambridge (UK). \\ 
M37 & Aur & Opn CL & 15x, 30x & TV60F6, Newcastle, Cambridge (UK). \\ 
M38 & Aur & Opn CL & 15x, 30x & TV60F6, Newcastle, Cambridge (UK). \\ 
M41 & CMa & Opn CL & 15x & TV60F6, Cambridge (UK). \\ 
M42 & Ori & CL+Neb & 36x; 15x, 18x, 30x, 36x, 51x, 72x & Orion nebula. C114F8, Venice (IT); TV60F6, Newcastle, Cambridge (UK). \\ 
M43 & Ori & Brt Neb & 36x; 15x, 18x, 30x, 36x, 51x, 72x & Detached part of Orion Nebula. C114F8, Venice (IT); TV60F6, Newcastle, Cambridge (UK). \\ 
M44 & Cnc & Opn CL & 15x & Beehive Cluster. B15x70, TV60F6, Newcastle (UK). \\ 
M45 & Tau & CL+Neb & 36x; 15x, 51x & Pleiades. C114F8, Venice (IT); TV60F6, Newcastle, Cambridge (UK). \\ 
M57 & Lyr & Pln Neb & 37x; 15x & Ring Nebula. C114F8, Venice, Lorenzago (IT); B15x70, Newcastle (UK). \\ 
M67 & Cnc & Opn CL & 15x & The King Cobra Cluster. TV60F6, Cambridge (UK). \\ 
M81 & UMa & Galaxy & 15x & Bode's Galaxy. B15x70, Newcastle (UK). \\ 
M82 & UMa & Galaxy & 15x & Cigar Galaxy. B15x70, Newcastle (UK). \\ 
M92 & Her & Glob CL & 15x & B15x70, Newcastle (UK). \\ 
M110 & And & Galaxy & 15x & M31 Companion. B15x70, Luton, Devon (UK). \\ 
C14 - NGC869 / 884 & Per & Opn CL & 15x, 51x & Double Cluster. B15x70, Exeter, Newcastle (UK); TV60F6, Newcastle (UK). \\ 
C28 - NGC752 & And & Opn CL & 15x & B15x70, Newcastle (UK). \\ 
C37 - NGC6885 & Vul & Opn CL & 15x & B15x70, Newcastle (UK). \\ 
C39 - NGC2392 & Gem & Pln Neb & 15x, 51x, 72x & Eskimo Nebula. TV60F6, Cambridge (UK). \\ 
C41 & Tau & Opn CL & 36x; 15x, 18x, 51x & Hyades. C114F8, Venice (IT); B15x70, TV60F6, Newcastle (UK). \\ 
C49 - NGC2237 & Mon & Neb & 15x & Rosette Nebula. TV60F6, Cambridge (UK). \\ 
C50 - NGC2244 & Mon & Opn CL & 15x & Satellite cluster. TV60F6, Newcastle, Cambridge (UK). \\ 
Stock1 & Vul & Opn CL & 15x & B15x70, Newcastle (UK). \\ 
Stock2 & Cas & Opn CL & 15x & B15x70, TV60F6, Newcastle (UK). \\ 
Mel20 & Per & Opn CL & 15x & B15x70, TV60F6, Newcastle (UK). \\ 
Mel25 & Tau & Opn CL & 36x; 15x, 51x & Hydes. C114F8, Venice (IT); TV60F6, Newcastle, Cambridge (UK). \\ 
Mel111 & CmB & Opn CL & 15x & Coma Berenices Star Cluster. B15x70, Newcastle (UK). \\ 
Cr39 & Ori & Opn CL & 36x; 15x, 51x & TV60F6, Cambridge (UK). \\ 
Cr65 & Tau & Opn CL & 15x & TV60F6, Cambridge (UK). \\ 
Cr68 & Ori & Opn CL & 15x, 30x & TV60F6. Newcastle, Cambridge (UK). \\ 
Cr89 & Gem & Opn CL & 15x & TV60F6. Newcastle, Cambridge (UK). \\ 
Cr97 & Mon & Opn CL & 15x, 30x & TV60F6, Cambridge (UK). \\ 
Cr106 & Mon & Opn CL & 15x, 30x & TV60F6, Cambridge (UK). \\ 
Cr107 & Mon & Opn CL & 15x, 30x & TV60F6, Cambridge (UK). \\ 
Cr399 & Vul & Opn CL & 15x & Brocchi's Cluster, the Coathanger. B15x70, Newcastle (UK). \\ 
NGC1750 & Tau & Opn CL & 15x, 51x & TV60F6, Cambridge (UK). \\ 
NGC1980 & Ori & Neb & 15x & TV60F6, Cambridge (UK). \\ 
NGC1981 & Ori & Opn CL & 15x, 51x & TV60F6, Cambridge (UK). \\ 
NGC2264 & Mon & Opn CL & 36x; 15x, 51x & C114F8, Venice (IT); TV60F6, Cambridge (UK). \\ 
NGC6633 & Oph & Opn CL & 36x; 15x & C114F8, Venice (IT); B15x70, Newcastle (UK). A beautiful chain of stars. \\ 
IC4756 & Ser & Opn CL & 36x; 15x & Graff's Cluster. C114F8, Venice (IT); B15x70, Newcastle (UK). \\ 
27 Hydra & Hya & Dbl Star & 51x & TV60F6, Cambridge (UK). \\ 
Gamma & UMi & Dbl Star & 51x & Pherkad. TV60F6, Cambridge (UK). \\ 
Beta & And & Dbl Star & 15x & Mirach. B15x70, Newcastle (UK). \\ 
Alpha & Her & Dbl Star & 15x & Rasalgethi. B15x70, Newcastle (UK). \\ 
71-72 & Oph & Dbl Star & 15x & B15x70, Newcastle (UK). \\ 
Alpha & Leo & Dbl Star & 37x, 74x & Regulus. C114F8, Italy (IT). \\ 
Beta & Ori & Dbl Star & 37x, 74x; 15x & Rigel. C114F8, Italy (IT), B15x70, Newcastle (UK). \\ 
Alpha & UMa & Dbl Star & 15x; 30x & Dubhe. B15x70, TV60F6, Newcastle (UK). \\ 
Zeta & UMa & Dbl Star & 15x; 15x, 51x & Mizar and Alcor. B15x70, TV60F6, Newcastle, Cambridge (UK). \\ 
19-20 & Dra & Dbl Star & 51x & TV60F6, Cambridge (UK). \\ 
Eta-HIP80309A & Dra & Dbl Star & 51x & TV60F6, Cambridge (UK). \\ 
Beta & Cyg & Dbl Star & 72x & Albireo. C114F8, Venice (IT); TV60F6, Newcastle (UK). \\ 
Sun &  & Star & 50x; 15x, 51x & TV60F6, Cambridge (UK). \\ 
Moon &  & Satellite & 36x, 90x, 180x; 15x, 51x, 72x, 103x, 144x & C114F8, Italy (IT); B15x70, TV60F6, Newcastle, Cambridge (UK). \\ 
Mercury &  & Planet & 15x & B15x70, Newcastle (UK). \\ 
Venus &  & Planet & 36x; 72x, 144x & C114F8, Venice (IT); TV60F6, Cambridge (UK). \\ 
Mars &  & Planet & 36x, 90x,180x & C114F8, Venice (IT). \\ 
Jupiter &  & Planet & 36x, 90x, 180x; 72x, 103x, 144x & C114F8, Italy (IT); B15x70, TV60F6, Newcastle, Cambridge (UK). \\ 
Saturn &  & Planet & 36x, 90x, 180x & C114F8, Venice (IT). \\ 
Uranus &  & Planet & 10x; 15x; 15x & B10x50, Lorenzago (IT); B15x70, TV60F6, Newcastle (UK). \\ 
\hline 
\end{longtable} 

\end{document}
