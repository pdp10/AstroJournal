\begin{tabular}{ p{1.7in} p{1.2in} p{1.5in} p{4.2in}}
{\bf Date:} & 11/06/2015 & {\bf Telescopes:} & Tele Vue 60 F6 \\ 
{\bf Time:} & 10.00pm-12.00pm & {\bf Eyepieces:} & TV Panoptic 24mm, Nagler 13mm, Nagler 3.5mm, Bresser SA 2x \\ 
{\bf Location:} & Cambridge, UK & {\bf Power, Exit pupil, FOV:} & 15x, 4mm, 4.30deg; 28x, 2.2mm, 2.80deg; 103x, 0.6mm, 0.77deg; 206x, 0.3mm, 0.38deg \\ 
{\bf Altitude (MAMSL):} & 12m & {\bf Filters:} &  \\ 
{\bf Temperature (C):} & 15 (wind: 2km/h) & & \\ 
{\bf Seeing (Antoniadi scale):} & 2 - Slight undulations & & \\ 
{\bf Transparency (AAAA scale):} & 3 - Somewhat clear & & \\ 
\end{tabular}
% Detailed observation data
\centering 
\begin{longtable}{ p{0.8in}  p{0.3in}  p{0.5in}  p{0.9in}  p{5.8in} }
\hline 
{\bf Target} & {\bf Cons} & {\bf Type} & {\bf Power} & {\bf Notes} \\ 
\hline 
Saturn & Lib & Planet & 103x, 206x & Very nice view of Saturn tonight. At both 103x and 206x, the Cassini division was detectable when the sky appeared steady for few seconds. It appeared as a soft grey shade on the lateral parts of the rings. This was not always visible, but just for few seconds when the seeing was steady and no wind blew. Titan was visible too. A band on the planet appeared as a soft darker gradient compared to the planet colour. \\ 
Venus & Gem & Planet & 103x, 206x & Phase about 50\%. No detail visible, but the image was sufficiently stable. A SPF would have helped, but I forgot it at home.  \\ 
Jupiter & Leo & Planet & 103x, 206x & 206x was too much for Jupiter tonight. Mostly seen it at 103x. Three moons visible, whereas the fourth seemed behind the planet. North and South Hemisphere bands visible. On the North Hemisphere another band was also detectable. No GRS visible. \\ 
NGC6992 / NGC6960 - C33 / C34 & Cyg & Glob CL & 15x +/- OIII, 28x +/- OIII & Veil Nebula. No visible or detectable. I carefully searched the stars and positioned at 52 Cygni. 28x +OIII seemed to show a soft transparent cloud, but I cannot say that that was the Veil Nebula. The sky was not fully transparent and dark. This might be the reason. \\ 
Epsilon Lyr & Lyr & Dbl star & 103x & Double Double. Just managed to see the two pairs, although the separation was not clear. It was just a tiny more bare elongated stars. I am not sure, but I suspect this was more due to the Nagler 3.5mm. Next time I will test it with the Vixen 5mm which to me is better. \\ 
Beta1 Sco & Sco & Dbl star & 103x & It was clearly split, but the seeing was not very nice near the horizon and the stars light was not puntiform. The small star is blue. This double star is the top of the three stars of Scorpio. \\ 
\hline 
\end{longtable} 
